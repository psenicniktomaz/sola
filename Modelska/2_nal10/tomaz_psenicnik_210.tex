\documentclass[12pt,a4paper]{article}

\usepackage[utf8x]{inputenc}   % omogoča uporabo slovenskih črk kodiranih v formatu UTF-8
\usepackage[slovene]{babel}    % naloži, med drugim, slovenske delilne vzorce

\usepackage{subcaption}
\usepackage[hyphens]{url}
\usepackage{hyperref}


\usepackage[pdftex]{graphicx}
\usepackage{wrapfig}

\usepackage{amsmath}
%\renewcommand{\vec}[1]{\boldsymbol{#1}} %naredi vector kot bold zapis

\usepackage{float}

\usepackage{amssymb}

%\documentclass[a4paper, 12pt]{article}
%\usepackage[slovene]{babel}
%\usepackage[latin2]{inputenc}
%\usepackage[T1]{fontenc}
%\usepackage{makeidx}%za stvarno kazalo
%\makeindex%naredi stvarno kazalo
%\usepackage{tikz}% paket za kroge

\title{\textbf{Modelska analiza 2} \\ 10. naloga - Direktno reševanje Poissonove enačbe \\}
	\author{Študent: Pšeničnik Tomaž}
	
	


	
\begin{document}

\pagenumbering{gobble}

	\begin{figure} [h]
  \centering
  \includegraphics[width=12 cm]{logo_fmf.png}
  \maketitle
\end{figure}
	
	
	
	\newpage
	\pagenumbering{arabic}
	
	

\section*{Poves kvadratne opne}

Tokratna naloga je najti poves kvadratne opne, ki je obtežena kot prikazuje slika \ref{fig:slika1}. Problem že znamo rešiti z iteracijsko \textit{Jacobijevo} metodo ali podobnim postopkom in nato prevesti na postopek pospešene relaksacije, kot v nalogi 5. Na hitro se spomnimo \textit{Jacobijeve} iteracije:
\begin{equation} \label{eq:enacba1}
 u_{j,k} ^{n+1}= \left[h^{2}g_{j,k} + u_{j+1,k}^{n} + u_{j-1,k}^{n} + u_{j,k+1}^{n} + u_{j,k-1}^{n} \right]/4,
\end{equation}
kjer je $h$ razmik med dvema sosednjima točkama, $g_{j,k}$ pa obtežitev območja v dani točki. Razmik $h$ privzamemo, da je enak med vsemi sosednjimi točkami.

\begin{figure}[H]
\begin{center}
\includegraphics[width=0.3\textwidth]{lik2.pdf}
\caption{Obtežitev po območju} \label{fig:slika1}
\end{center}
\end{figure} 



Toda tokratno nalogo lahko rešimo z bolj učinkovito metodo, in sicer s \textit{Fourierovo} transformacijo oz. FFT (\textit{Fast Fourier Transform}). \textit{Fourierovo} metodo lahko uporabimo vedno takrat, kadar so meje področja koordinatne črte katerega od separabilnih koordinatnih sistemov (npr. pravokotnik), robni problemi pa linearni. V tokratnem primeru, ko gre za kvadrat, razvijemo območje in odmike z 2D \textit{Fourierovo} transformacijo:
\begin{align}
u_{j,k}= \sum_{j=0,k=0}^{N-1} U^{m,n} e^{2\pi i mj/N} e^{2\pi i nk/N}, \label{eq:enacba2} \\
g_{j,k}=\sum_{j=0,k=0}^{N-1}  G^{m,n} e^{2\pi i mj/N} e^{2\pi i nk/N},\label{eq:enacba3}
\end{align}
kjer so $u$ odmiki, $g$ pa funkcija obtežitve. Z vstavitvijo enačb (\ref{eq:enacba2}) in (\ref{eq:enacba3}) v enačbo (\ref{eq:enacba1}), pridemo do koristne zveze
\begin{equation}\label{eq:enacba4}
U^{m,n}=\frac{h^{2} G^{m,n}}{2 \cos(2\pi m /N)\cos(2\pi n/N)-4},
\end{equation}
katera pa ne velja za $m=n=0$. Ker morajo biti vsi odmiki v našem primeru na robovih nič, sledi
\begin{align}
u_{0,0}=0 = \sum_{m=0,j=0}^{N-1} U^{m,n} \rightarrow U^{0,0}= -\sum_{\textrm{ostali}} U^{m,n}. 
\end{align}
Rešitev enakomerno obtežene kvadratne opne poznamo že iz naloge 5. Zato si najprej poglejmo rešitev s \textit{Fourierovo} metodo. Da zadostimo robnim pogojem ($u=0$ na robu), je potrebno območje  liho razširiti v obe strani. Razširitev je prikazana na sliki \ref{fig:slika2}
\begin{figure}[H]
\begin{center}
\includegraphics[width=0.5\textwidth]{2dFFT_brezutezi_lik.pdf}
\caption{Obtežitev po območju.} \label{fig:slika2}
\end{center}
\end{figure} 
\noindent Tako dobimo že znano rešitev, prikazano na slikah \ref{fig:slika3}. Pri obratni transformaciji $U\rightarrow u$ dobimo skoraj povsem realne rešitve, imaginarni deli so zanemarljivo majhni.
\begin{figure}[H]
    \centering
    \begin{subfigure}[b]{0.5\textwidth}  			
        \includegraphics[width=\textwidth]{2dFFT_brezutezi_1.pdf}
    \end{subfigure}
    \begin{subfigure}[b]{0.45\textwidth}  			
        \includegraphics[width=\textwidth]{2dFFT_brezutezi_2.pdf}     
    \end{subfigure}
    \caption{Na levi rešitvi so prikazani le realni deli, na desni rešitvi so prikazani absolutni odmiki leve rešitve zgornje desne slike.} \label{fig:slika3}
\end{figure}

\noindent Naredimo hitrostno primerjavo med obema metodama. Kako se dokopati do števila iteracij v knjižici \texttt{scipy.fftpack.fft2} nam žal ni uspelo, zato si poglejmo le časovno primerjavo med metodami. Seveda je potrebno vzeti v zakup, da je število iteracij pri metodi SOR toliko večje, kolikor je zahtevana natančnost. Rezultate prikazujeta sliki \ref{fig:slika10}.

\begin{figure}[H]
    \centering
    \begin{subfigure}[b]{0.45\textwidth}  			
        \includegraphics[width=\textwidth]{time_comparance1.pdf}
    \end{subfigure}
    \begin{subfigure}[b]{0.45\textwidth}  			
        \includegraphics[width=\textwidth]{time_comparance2.pdf}     
    \end{subfigure}
    \caption{Časovna primerjava metod levo in primerjava s številom iteracij med \textit{Jacobijevo} iteracisjko metodo in SOR  desno.} \label{fig:slika10}
\end{figure}

Sedaj se lotimo reševanja različno obteženega območja. Za referenco najprej rešimo z metodo SOR. Območje razdelimo, kot je prikazano na sliki \ref{fig:slika4}. \begin{figure}[H]
\begin{center}
\includegraphics[width=0.3\textwidth]{obmocje.pdf}
\caption{Točke podane po območju, obkrožene točke so notranje točke območja, po katerih računamo.} \label{fig:slika4}
\end{center}
\end{figure}
\noindent Tako bomo uporabili malenkost modificirano enačbo \ref{eq:enacba1} in sicer:
\begin{itemize}
\item Vogalne točke (obkrožene zeleno - so štiri): 
\begin{equation*}
\tilde{u}_{j,k} ^{n}= u_{j,k} ^{n+1}= \left[h^{2}g_{j,k} + u_{j+1,k}^{n} + u_{j-1,k}^{n} + u_{j,k+1}^{n} + u_{j,k-1}^{n} \right]/6,
\end{equation*}
\item Robne točke (obkrožene z modro): 
\begin{equation*}
\tilde{u}_{j,k} ^{n}= u_{j,k} ^{n+1}= \left[h^{2}g_{j,k} + u_{j+1,k}^{n} + u_{j-1,k}^{n} + u_{j,k+1}^{n} + u_{j,k-1}^{n} \right]/5,
\end{equation*}
\item Vse ostale - notranje točke (obkrožene z rdečo):
\begin{equation*}
\tilde{u}_{j,k} ^{n}= u_{j,k} ^{n+1}= \left[h^{2}g_{j,k} + u_{j+1,k}^{n} + u_{j-1,k}^{n} + u_{j,k+1}^{n} + u_{j,k-1}^{n} \right]/4.
\end{equation*}
\end{itemize}
Računali bomo le po notranjih točkah območja. Zunanje točke odmikov fiksiramo $u=0$.
 Računali bomo po območju prikazanem na slikah \ref{fig:slika5}.

\begin{figure}[H]
    \centering
    \begin{subfigure}[b]{0.45\textwidth}  			
        \includegraphics[width=\textwidth]{sor_obtez_1_lik.pdf}
    \end{subfigure}
    \begin{subfigure}[b]{0.45\textwidth}  			
        \includegraphics[width=\textwidth]{sor_obtez_2_lik.pdf}     
    \end{subfigure}
    \caption{Prvi primer levo in drugi primer desno.} \label{fig:slika5}
\end{figure}

\noindent Tako dobimo referenčni rešitvi prikazani na slikah \ref{fig:slika6}.
\begin{figure}[H]
    \centering
    \begin{subfigure}[b]{0.45\textwidth}  			
        \includegraphics[width=\textwidth]{sor_obtez_1.pdf}
    \end{subfigure}
    \begin{subfigure}[b]{0.45\textwidth}  			
        \includegraphics[width=\textwidth]{sor_obtez_2.pdf}     
    \end{subfigure}
    \caption{Prvi primer levo in drugi primer desno za 100x100 veliko bomočje.} \label{fig:slika6}
\end{figure}

Najprej si poglejmo, kako razširiti območje za \textit{Fourierovo} transformacijo. Razširitev območja je prikazana na sliki \ref{fig:slika7}. Območju smo dodali še zunanje točke, kateri smo pripisali vrednost obteženega območja na tej meji.

\begin{figure}[H]
    \centering
    \begin{subfigure}[b]{0.45\textwidth}  			
        \includegraphics[width=\textwidth]{2dFFT_utez_1_lik.pdf}
    \end{subfigure}
    \begin{subfigure}[b]{0.45\textwidth}  			
        \includegraphics[width=\textwidth]{2dFFT_utez_2_lik.pdf}     
    \end{subfigure}
    \caption{Prvi primer levo in drugi primer desno za 21x21 veliko območje enega kvadranta.} \label{fig:slika7}
\end{figure}

\noindent 2D \textit{Fourierovo} transformacijo poženemo po celotnem območju. Rezultata sta prikazana na slikah \ref{fig:slika8} in \ref{fig:slika9}.

\begin{figure}[H]
    \centering
    \begin{subfigure}[b]{0.5\textwidth}  			
        \includegraphics[width=\textwidth]{2dFFT_utez_1_1.pdf}
    \end{subfigure}
    \begin{subfigure}[b]{0.45\textwidth}  			
        \includegraphics[width=\textwidth]{2dFFT_utez_1_2.pdf}     
    \end{subfigure}
    \caption{Prvi primer za 201x201 veliko območje enega kvadranta.} \label{fig:slika8}
\end{figure}

\begin{figure}[H]
    \centering
    \begin{subfigure}[b]{0.5\textwidth}  			
        \includegraphics[width=\textwidth]{2dFFT_utez_2_1.pdf}
    \end{subfigure}
    \begin{subfigure}[b]{0.45\textwidth}  			
        \includegraphics[width=\textwidth]{2dFFT_utez_2_2.pdf}     
    \end{subfigure}
    \caption{Drugi primer za 201x201 veliko območje enega kvadranta.} \label{fig:slika9}
\end{figure}

Poglejmo si še princip kombinirane metode, pri kateri v eni smeri uporabimo sinusno FFT, v drugi smeri pa SOR. Metoda nam bo prišla prav v nadaljevanju, zato je koristno preveriti pravilnost izračunov na že znanih rezultatih za poves opne. Najprej potrebujemo območje, na katerem bomo izvedli račune. Primera območja prikazujeta sliki \ref{fig:slika12}
\begin{figure}[H]
    \centering
    \begin{subfigure}[b]{0.45\textwidth}  			
        \includegraphics[width=\textwidth]{kombinirana_obmocje.pdf}
    \end{subfigure}
    \begin{subfigure}[b]{0.45\textwidth}  			
        \includegraphics[width=\textwidth]{kombinirana_2_obmocje.pdf}
    \end{subfigure}
    \caption{Območji za kvadratno opno brez obtežitve in z obtežitvijo.} \label{fig:slika12}
\end{figure}
\noindent Tokrat postavimo rob območja na 0, ker ne želimo odmika na robu.
V navpični smeri bomo naredili sinusno FFT območja:
\begin{align} \label{eq:enacba7}
G^{m} _{ \ \ k} = \sum_j g_{j,k}2\sin(\pi j/N).
\end{align}
Če razvijemo še odmike po sinusni FFT:
\begin{align}\label{eq:enacba8}
U^{m} _{ \ \ k } =  \sum_j u_{j,k} 2 \sin(\pi j/N)
\end{align}
in enačbi (\ref{eq:enacba7}) in (\ref{eq:enacba8}) vstavimo v znano zvezo (\ref{eq:enacba1}), dobimo iteracijsko zvezo v \textit{Fourierovem} prostoru:
\begin{equation}
(U^{m} _{ \ \ k })^{n+1}= \left[(U^{m} _{ \ \ k-1 })^{n} +(U^{m} _{ \ \ k+1 })^{n} +(G^{m} _{ \ \ k })^{n} h^{2} \right] / (4-2\cos(m \pi /N)),
\end{equation}
s katero računamo po vseh vrsticah. Na koncu naredimo le še obratno sinusno FFT:
\begin{equation}
u_{j,k} = \sum _m U^{m} _{ \ \ k } \sin(m\pi j /N)/N
\end{equation}
in dobimo že znane rezultate prikazane na slikah \ref{fig:slika13}
\begin{figure}[H]
    \centering
    \begin{subfigure}[b]{0.32\textwidth}  			
        \includegraphics[width=\textwidth]{kombinirana.pdf}
    \end{subfigure}
    \begin{subfigure}[b]{0.32\textwidth}  			
        \includegraphics[width=\textwidth]{kombinirana_2.pdf}
    \end{subfigure}
    \begin{subfigure}[b]{0.32\textwidth}  			
        \includegraphics[width=\textwidth]{kombinirana_3.pdf}
    \end{subfigure}
    \caption{Poves kvadratne opne dobljen s kombinirano metodo.} \label{fig:slika13}
\end{figure}

\section*{Temperaturni profil enakostraničnega valja}


Spomnimo se \textit{Jacobijeve} metode za cilindrične koordinate iz naloge 5:
\begin{equation}\label{eq:enacba6}
u_{j,k}^{n+1}=\left( g_{j,k} + \frac{u_{j,k-1}^{n} + u_{j,k+1}^{n}}{h_r ^{2}} + \frac{1}{r_k}\left(  \frac{u_{j,k+1}^{n} - u_{j,k-1}^{n}}{2h_r}\right) + \frac{u_{j-1,k}^{n} + u_{j+1,k}^{n}}{h_z ^{2}}\right)/d,
\end{equation}
kjer je
\begin{equation*}
d=\left( 
\frac{2}{h_r^{2}}+ \frac{2}{h_r ^{2}}  \right).
\end{equation*}
Dobimo referenčno rešitev s to metodo, prikazano na sliki \ref{fig:slika11}, kjer valj na osnovnicah hladimo pri temperaturi 0, na straneh pa grejemo pri temperaturi 1.

\begin{figure}[H]
    \centering
    \begin{subfigure}[b]{0.5\textwidth}  			
        \includegraphics[width=\textwidth]{valj_sor.pdf}
    \end{subfigure}
    \caption{Enakostraničen valj za 100x100 veliko območje.} \label{fig:slika11}
\end{figure}


Postopek lahko pospešimo, če v $z$ smeri naredimo sinusno FFT, uporabimo \textit{Jacobijevo} iteracijsko metodo v smeri $r$ in naredimo inverzno sinunso FFT v $z$ smeri. V enačbo (\ref{eq:enacba6}) vstavimo sinusne FFT transformiranke (\ref{eq:enacba7}) in (\ref{eq:enacba8}), ter ponovno dobimo iteracijsko zvezo s poljubno dolžino med vsemi sosednjimi točkami v $r$ smeri $h_r$ in $z$ smeri $h_z$:
\begin{align} \label{eq:enacba9}
&(U^{m} _{ \ \ k })^{n+1}  (d - \frac{1}{h_z^{2}}2\cos(\pi m /N))= \nonumber \\
&= (G^{m} _{ \ \ k}) ^{n} + (U^{m} _{ \ \ k-1 })^{n} \left(\frac{1}{h_r ^{2}} - \frac{1}{2h_r r_k} \right) + (U^{m} _{ \ \ k+1 })^{n} \left(\frac{1}{h_r ^{2}} + \frac{1}{2h_r r_k} \right).
\end{align}
V tokratnem primeru rešujemo \textit{Laplaceovo} enačbo
\begin{align*}
\nabla ^{2} u = 0,
\end{align*}
tako bo območje povsod nič. Odmike na plašču bomo fiskirali na ena, kot prikazuje slika \ref{fig:slika14} in tako dobili sinusno FFT le za dva stolpca.
\begin{figure}[H]
    \centering
    \begin{subfigure}[b]{0.5\textwidth}  			
        \includegraphics[width=\textwidth]{valj_kombinirana_3_obmocje.pdf}
    \end{subfigure}
    \caption{Začetni pogoji za enakostraničen valj s konstantno temperaturo na plašču.} \label{fig:slika14}
\end{figure}
\noindent Za izračun naslednjih $U^{m} _{ \ \ k }$, uporabimo dobljeno zvezo \ref{eq:enacba9}. Dobljeni rezultat prikazuje slika \ref{fig:slika15}
\begin{figure}[H]
    \centering
    \begin{subfigure}[b]{0.5\textwidth}  			
        \includegraphics[width=\textwidth]{valj_kombinirana_iteracija.pdf}
    \end{subfigure}
    \caption{Dobljeni $U^{m} _{ \ \ k }$. } \label{fig:slika15}
\end{figure}

\noindent Preostane nam samo še inverzna sinusna FFT in dobimo znan rezultat, prikazan na sliki \ref{fig:slika16}.

\begin{figure}[H]
    \centering
    \begin{subfigure}[b]{0.5\textwidth}  			
        \includegraphics[width=\textwidth]{valj_kombinirana.pdf}
    \end{subfigure}
    \caption{Temperaturni profil valja z višjo temperaturo na osnovnicah.} \label{fig:slika16}
\end{figure}

Če želimo imeti na plašču temperaturo nič in na osnovnicah temperaturo 1, je vse kar moramo storiti, da zamenjamo enice z -1 na sliki \ref{fig:slika14} in h končnemu rezultatu prištejemo ena. Na tak način lahko dobimo tudi poljubno temperaturo na plašču in osnovnicah. Primer je prikazan na sliki \ref{fig:slika17}.

\begin{figure}[H]
    \centering
    \begin{subfigure}[b]{0.5\textwidth}  			
        \includegraphics[width=\textwidth]{valj_kombinirana_2.pdf}
    \end{subfigure}
    \caption{Temperaturni profil valja z višjo temperaturo na plašču.} \label{fig:slika17}
\end{figure}



\end{document}
