\documentclass[12pt,a4paper]{article}

\usepackage[utf8x]{inputenc}   % omogoča uporabo slovenskih črk kodiranih v formatu UTF-8
\usepackage[slovene]{babel}    % naloži, med drugim, slovenske delilne vzorce

\usepackage{subcaption}
\usepackage[hyphens]{url}
\usepackage{hyperref}


\usepackage[pdftex]{graphicx}
\usepackage{wrapfig}

\usepackage{amsmath}
%\renewcommand{\vec}[1]{\boldsymbol{#1}} %naredi vector kot bold zapis

\usepackage{float}

\usepackage{amssymb}

%\documentclass[a4paper, 12pt]{article}
%\usepackage[slovene]{babel}
%\usepackage[latin2]{inputenc}
%\usepackage[T1]{fontenc}
%\usepackage{makeidx}%za stvarno kazalo
%\makeindex%naredi stvarno kazalo
%\usepackage{tikz}% paket za kroge

\title{\textbf{Modelska analiza 2} \\ 2. naloga - Navadne diferencialne enačbe: robni problem \\}
	\author{Študent: Pšeničnik Tomaž}
	
	


	
\begin{document}

\pagenumbering{gobble}

	\begin{figure} [h]
  \centering
  \includegraphics[width=12 cm]{logo_fmf.png}
  \maketitle
\end{figure}
	
	
	
	\newpage
	\pagenumbering{arabic}
	
	
	
\section*{Oblika vrteče se vrvice}

Tokratna naloga je izračunati  obliko vrteče vrvice, ki je pripeta na obeh koncih. Problem je prikazan na sliki \ref{fig:slika1}.

\begin{figure}[H]
    \centering
        \includegraphics[width=0.2\textwidth]{vrvica.pdf}
    \caption{Skica vrteče se vrvice.} \label{fig:slika1}
\end{figure}

\noindent Najprej si poglejmo, kako pridemo do seta enačb (\ref{eq:enacba1}), ki so znane za vrtečo se vrv:
\begin{align} \label{eq:enacba1}
\frac{d}{ds} \left(F\frac{dx}{ds} \right) + \rho \omega ^{2} x =& 0 \nonumber  \\
\frac{d}{ds} \left(F\frac{dy}{ds} \right) -\rho g =& 0  \nonumber \\
\left( \frac{dx}{ds} \right) ^{2} + \left( \frac{dy}{ds} \right) ^{2} =& 1.
\end{align}
Zamislimo si napet košček vrvi, prikazan na sliki \ref{fig:slika2}.

\begin{figure}[H]
    \centering
        \includegraphics[width=0.7\textwidth]{eq_struna.pdf}
    \caption{Delček vrvi.} \label{fig:slika2}
\end{figure}
\noindent Za košček vrvi dolžine $ds$ in z gostoto $\rho$ zapišimo Newtonov zakon $m a=F$ v $x$ in $y$ smeri:
\begin{align} \label{eq:enacba2}
ds \rho \frac{\partial ^{2} x}{\partial t^{2}} & =  (F(s +ds) - F(s)) \frac{dx}{ds} + \rho ds \omega ^{2} x \nonumber \\
ds \rho \frac{\partial ^{2} y}{\partial t^{2}} & =  (F(s +ds) - F(s)) \frac{dy}{ds} - ds \rho g.
\end{align}
Enačbi (\ref{eq:enacba2}) delimo z $ds$ in postavimo v stacionarno stanje ($\frac{d}{dt} \rightarrow 0)$. Na desni strani prepoznamo odvod po sili. Zapišemo samo še Pitagorov izrek (\ref{eq:enacba3})
\begin{equation} \label{eq:enacba3}
ds^{2}  = dx^{2} +dy^{2}
\end{equation}
in dobimo set enačb \ref{eq:enacba1}. S substitucijami $s \rightarrow s/l, \ \ x\rightarrow x/l, \ \ y \rightarrow y/l, \ \ F \rightarrow F/\rho gl, \ \ dx/ds = \cos \alpha, \ \ dy/ds = \sin \alpha, \ \ \beta = \omega ^{2} l/g$, če je $l$ dolžina vrvi, pretvorimo sistem (\ref{eq:enacba1} v sledeči sistem \ref{eq:enacba4}:
 \begin{align} \label{eq:enacba4}
 \frac{dx}{ds} &= \cos \alpha  &\qquad \frac{dy}{ds} & = \sin \alpha \nonumber \\ 
 \frac{dF}{ds}& = -\beta x \cos \alpha + \sin \alpha & \qquad \frac{d \alpha }{ds} &= \frac{1}{F} \left( \beta x \sin \alpha + \cos \alpha \right).
 \end{align}
Poznamo obesišči vrvice. Zgornje obesišče ima koordinate (0,0), spodnje obesišče pa postavimo na (0,$y_f$), pri čemer je $|y_f| < l$.

Računanja se bomo lotili s strelsko metodo. Reševali bomo sistem enačb (\ref{eq:enacba4}), pri čemer bomo pričeli integrirati v začetni točki ($x=0,y=0$) po celotni dolžini $l$ vrvice in poskušali najti ustrezne parametre, da je konec vrvice v točki ($x=0,y=y_f$. Določimo $\beta$ in vzamemo prvi približek naklona v začetni točki $\alpha _0$ in začetni približek sile $F_0$. Prva integracija (skoraj) zagotovo ne bo dala pravilne rešitve, zato bomo glede na prvo rešitev vzeli primeren naslednji približek in ponavljali postopek, dokler metoda ne skonvergira.

Metoda ni delovala za vse $\beta$, in sicer za $\beta \lesssim 2.5$. Za zelo visoke $\beta$ metoda ni bila preizkušena. Poglejmo si za začetek nekatere oblike vrvice, z začetnim približkom naklona in sile $\alpha_0=0, \ \ F_0=1$. Rešitve so prikazane na slikah \ref{fig:slika3}. Kot vidimo na slikah \ref{fig:slika3}, dobimo za male $\beta$ intuitivno rešitev, za večje $\beta$ pa rešitve delujejo kot neke vrste višji harmoniki... Te rešitve niso fizikalno smiselne, saj se vrteča se vrvica ne postavi v takšno obliko. Metoda  pač konvergira k eni izmed rešitvi enačb glede na začetni približek.
\begin{figure}[H]
    \centering
    \begin{subfigure}[b]{0.45\textwidth}  			
        \includegraphics[width=\textwidth]{oblika_vrvi.pdf}
    \end{subfigure}
    \begin{subfigure}[b]{0.45\textwidth}  			
        \includegraphics[width=\textwidth]{vrvica_4.pdf}
    \end{subfigure}
    \caption{Oblika vrvi za začetni približek naklona in sile $\alpha_0=0, \ \ F_0=1$. } \label{fig:slika3}
\end{figure} 
\noindent Pogledamo si lahko maksimalen odmik in silo, ki deluje na vpeto vrvico v točki ($x=0,y=0$). Maksimalen odmik in silo glede na izbrano $\beta$ prikazuje slika \ref{fig:slika4}.

\begin{figure}[H]
    \centering
    \begin{subfigure}[b]{0.45\textwidth}  			
        \includegraphics[width=\textwidth]{vrvica_5.pdf}
    \end{subfigure}
    \begin{subfigure}[b]{0.45\textwidth}  			
        \includegraphics[width=\textwidth]{vrvica_5_sila.pdf}
    \end{subfigure}
    \caption{Največji odmik vrvice levo in potek sile $F_0$ v odvisnosti od $\beta$.} 
    \label{fig:slika4}
\end{figure}

\noindent Odmik se z večanjem beta povečuje, pri $\beta \sim 20$ dobimo prvi višji harmonik, sledi ponoven skok, kjer vrvica prevzame podobno obliko za $\beta=12$ na sliki \ref{fig:slika3}. Sledi ponoven skok, kjer je odmik praktično konstanten in rahlo pada z večanjem $\beta$. V tem režimu vrvica prevzame obliko nekega harmonika. Enako se dogodi pri grafu za $F_0$.

Stvari smo si pogledali vedno za nov isti začetni približek. Drugače je, če za začetni približek naslednje $\beta$ vzamemo rezultat $\alpha_0$ in $F_0$ prejnšje $\beta$. Takrat ne dobimo več harmonikov, temveč le eno obliko vrvi, kar je fizikalno smiselen rezultat. Takšen rezultat prikazuje slika \ref{fig:slika5}.

\begin{figure}[H]
    \centering
        \includegraphics[width=0.5\textwidth]{vrvica_7.pdf}
    \caption{Oblika vrvi.} \label{fig:slika5}
\end{figure}

\noindent Tudi rezultata za maksimalen odmik in silo sta pričakovana. Ogledamo si ju lahko na slikah \ref{fig:slika6}.

\begin{figure}[H]
    \centering
    \begin{subfigure}[b]{0.45\textwidth}  			
        \includegraphics[width=\textwidth]{vrvica_8.pdf}
    \end{subfigure}
    \begin{subfigure}[b]{0.45\textwidth}  			
        \includegraphics[width=\textwidth]{vrvica_8_sila.pdf}
    \end{subfigure}
    \caption{Največji odmik vrvice levo in potek sile $F_0$ v odvisnosti od $\beta$.} 
    \label{fig:slika6}
\end{figure}

\section*{H\'enon in Heiles-ov potencial}
Pri študiju gibanja zvezd skozi galaksijo sta H\'enon in Heiles vpeljala 3-števno simetrični potencial:
\begin{equation} \label{eq:enacba5}
U(x,y)=\frac{1}{2}(x^{2} +y^{2}) + x^{2}y -\frac{1}{3}y^{3}.
\end{equation}
Na sliki \ref{fig:slika7} je prikazan potencial (\ref{eq:enacba5}) z nekaj ekvipotencialnimi krivuljami. Gibanje v potencialu je omejeno za energije $E < 1/6$. Tako se bomo skoncentrirali le na gibanje v tem območju, ki je na sliki \ref{fig:slika7} omejeno s trikotnikom.
\begin{figure}[H]
    \centering
        \includegraphics[width=0.7\textwidth]{potencial_1.pdf}
    \caption{Potencial (\ref{eq:enacba5}) in nekaj ekvipotencialnih krivulj.} \label{fig:slika7}
\end{figure}
\noindent Zapišemo lahko Hamiltonian za naš sistem:
\begin{equation} \label{eq:enacba6}
H=T+U \rightarrow H= \frac{1}{2}(\dot{x}^{2} +\dot{y}^{2}) +U
\end{equation}
in vpeljemo spremenljivki $\dot{x}=u$ in $\dot{y}=v$. Iz Hamiltonijana zapišemo sistem enačb oz. enačbe gibanja:
\begin{align} \label{eq:enacba7}
\dot{x}&=u=\frac{\partial H}{\partial u} &\qquad \dot{y} &= v=\frac{\partial H}{\partial v} \nonumber \\
\dot{u}&= -\frac{\partial H}{\partial x}= -x -2xy & \qquad \dot{v}&=-\frac{\partial H}{\partial y}= -y -x^{2} +y^{2}
\end{align}

Podobno kot v prejšnji nalogi z vrvico, se bomo iskanja periodičnih orbital lotili s strelsko metodo, le da imamo edini pogoj konstantno energijo. Za začetni pogoj izberemo točko, v našem primeru bo to vedno $x_0=0$, $y_0$ pa bomo spreminjali, dokler ne bomo dobili zadovoljivega zadnjega  ali predzadnjega presečišča pri $x=0$ in $|y_0-y_p|< \epsilon$. Ko podamo začetno točko, imamo določeno kinetično energijo v tej točki, ni pa določene smeri hitrosti. Tako bomo hkrati s presečiščem pri $x=0$ iskali še začetno podano smer $\alpha_0$ tako, da bo veljalo $|\alpha_0 - \alpha_p| < \epsilon $ za dano presečišče. Pri vsakem iskanju trajektorije fiksiramo čas. Pri integraciji bomo vedno uporabili časovni korak $dt=0.001$ in sistemski $\epsilon$ pri iskanju ničel z modulom \texttt{scipy.optimize.root}.

Ob najdenih parametrih $y_0$ in $alpha_0$, je koristno pogledati, kako se sistem obnaša, če enačbe gibanja (\ref{eq:enacba7}) integriramo po zelo dolgem času. Opazimo, da nekatere dobljene rešitve niso primerne, saj imamo kvazi-periodično gibanje. Taka primera sta prikazana na slikah \ref{fig:slika8}. Črni piki na grafih, vedno prikazujeta začetno točko pri $x_0=0$ in končno točko po končanem integriranju do časa $t$.

\begin{figure}[H]
    \centering
    \begin{subfigure}[b]{0.45\textwidth}  			
        \includegraphics[width=\textwidth]{periodicne_orbite_12.pdf}
    \end{subfigure}
    \begin{subfigure}[b]{0.45\textwidth}  			
        \includegraphics[width=\textwidth]{periodicne_orbite_12_daljsa.pdf}
    \end{subfigure}
   
        \begin{subfigure}[b]{0.45\textwidth}  			
        \includegraphics[width=\textwidth]{periodicne_orbite_13.pdf}
    \end{subfigure}
    \begin{subfigure}[b]{0.45\textwidth}  			
        \includegraphics[width=\textwidth]{periodicne_orbite_13_daljsa.pdf}
    \end{subfigure}
    \caption{Kvazi-periodicni orbiti.} 
    \label{fig:slika8}
\end{figure}

Program pri različnih pogojih konvergira k različnim rešitvam. Zanimive so linearne orbite, prikazane na slikah \ref{fig:slika9}. Gibanje je omejeno do ekvipotencialne krivulje.
\begin{figure}[H]
    \centering
    \begin{subfigure}[b]{0.45\textwidth}  			
        \includegraphics[width=\textwidth]{periodicne_orbite_8.pdf}
    \end{subfigure}
    \begin{subfigure}[b]{0.45\textwidth}  			
        \includegraphics[width=\textwidth]{periodicne_orbite_8_daljsa.pdf}
    \end{subfigure}
   
        \begin{subfigure}[b]{0.45\textwidth}  			
        \includegraphics[width=\textwidth]{periodicne_orbite_10.pdf}
    \end{subfigure}
    \begin{subfigure}[b]{0.45\textwidth}  			
        \includegraphics[width=\textwidth]{periodicne_orbite_10_daljsa.pdf}
    \end{subfigure}
    
    \begin{subfigure}[b]{0.45\textwidth}  			
        \includegraphics[width=\textwidth]{periodicne_orbite_9.pdf}
    \end{subfigure}
    \begin{subfigure}[b]{0.45\textwidth}  			
        \includegraphics[width=\textwidth]{periodicne_orbite_9_daljsa.pdf}
    \end{subfigure}
    \caption{Linearne orbite.} 
    \label{fig:slika9}
\end{figure}
\noindent Največkrat je program skonvergiral k pogojem za orbito v obliki zaključene zanke. Oblika te orbite je prikazana na slikah \ref{fig:slika10}. Kot vidimo, nekega pravila ni, h kateri rešitvi program skonvergira. Za zadnji dve orbiti smo morali spremeniti začetne pogoje iskanja. Zadnje tri orbite so razvidni na desni sliki \ref{fig:slika10}.

\begin{figure}[H]
    \centering
    \begin{subfigure}[b]{0.45\textwidth}  			
        \includegraphics[width=\textwidth]{periodicne_orbite_krogi.pdf}
    \end{subfigure}
    \begin{subfigure}[b]{0.45\textwidth}  			
        \includegraphics[width=\textwidth]{periodicne_orbite_krogi_2.pdf}
    \end{subfigure}
    \caption{Zaključena zanka pri različnih konstantnih energijah. Začetni "guess" parametri so $y_0, \ \ \alpha_0$ za nek konstanten čas integracije $t$.} 
    \label{fig:slika10}
\end{figure}

Za konec si poglejmo še nekaj periodičnih orbit, ki smo jih dobili z naključnimi začetnimi parametri. Prikazane so na sliki \ref{fig:slika11}. Obstaja še več periodičnih orbit, ki pa jih s strelsko metodo nismo odkrili, oziroma, smo premalo preskušali različne začetne približke parametrov. Nekateri začetni približki $y_0$ so izven trikotnika na sliki potenciala \ref{fig:slika7}, za kar se ne bomo zmenili, saj je bil program spisan tako, da ustrezno preslika začetni pogoj v notranjost trikotnika.

\begin{figure}[H]
    \centering
    \begin{subfigure}[b]{0.45\textwidth}  			
        \includegraphics[width=\textwidth]{periodicne_orbite_1.pdf}
    \end{subfigure}
    \begin{subfigure}[b]{0.45\textwidth}  			
        \includegraphics[width=\textwidth]{periodicne_orbite_2.pdf}
    \end{subfigure}
   
    \begin{subfigure}[b]{0.45\textwidth}  			
        \includegraphics[width=\textwidth]{periodicne_orbite_3.pdf}
    \end{subfigure}
    \begin{subfigure}[b]{0.45\textwidth}  			
        \includegraphics[width=\textwidth]{periodicne_orbite_4.pdf}
    \end{subfigure}
    
    \begin{subfigure}[b]{0.45\textwidth}  			
        \includegraphics[width=\textwidth]{periodicne_orbite_6.pdf}
    \end{subfigure}
    \begin{subfigure}[b]{0.45\textwidth}  			
        \includegraphics[width=\textwidth]{periodicne_orbite_7.pdf}
    \end{subfigure}
    
        \begin{subfigure}[b]{0.45\textwidth}  			
        \includegraphics[width=\textwidth]{periodicne_orbite_11.pdf}
    \end{subfigure}
    \begin{subfigure}[b]{0.45\textwidth}  			
        \includegraphics[width=\textwidth]{periodicne_orbite_14.pdf}
    \end{subfigure}
    \caption{Nekaj periodičnih orbit} 
    \label{fig:slika11}
\end{figure}


\end{document}
