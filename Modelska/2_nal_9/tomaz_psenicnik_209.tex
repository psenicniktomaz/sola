\documentclass[12pt,a4paper]{article}

\usepackage[utf8x]{inputenc}   % omogoča uporabo slovenskih črk kodiranih v formatu UTF-8
\usepackage[slovene]{babel}    % naloži, med drugim, slovenske delilne vzorce

\usepackage{subcaption}
\usepackage[hyphens]{url}
\usepackage{hyperref}


\usepackage[pdftex]{graphicx}
\usepackage{wrapfig}

\usepackage{amsmath}
%\renewcommand{\vec}[1]{\boldsymbol{#1}} %naredi vector kot bold zapis

\usepackage{float}

\usepackage{amssymb}

%\documentclass[a4paper, 12pt]{article}
%\usepackage[slovene]{babel}
%\usepackage[latin2]{inputenc}
%\usepackage[T1]{fontenc}
%\usepackage{makeidx}%za stvarno kazalo
%\makeindex%naredi stvarno kazalo
%\usepackage{tikz}% paket za kroge

\title{\textbf{Modelska analiza 2} \\ 9. naloga - Metoda robnih elementov \\}
	\author{Študent: Pšeničnik Tomaž}
	
	


	
\begin{document}

\pagenumbering{gobble}

	\begin{figure} [h]
  \centering
  \includegraphics[width=12 cm]{logo_fmf.png}
  \maketitle
\end{figure}
	
	
	
	\newpage
	\pagenumbering{arabic}
	
	

\section*{Elektrostatsko polje nabitega traku }

Želimo najti elektrostatični potencial nabitega neskončno dolgega traku. S tem smo problem prevedli na 2D. Potencial takšnega traku širine $l$ nabitega z enotsko gostoto naboja, postavljenega v izhodišče koordinatnega sistema, opišemo z \textit{Greenovo} funkcijo $G_{0,r}=u(x,y)$:
\begin{align} \label{eq:enacba1}
u(x,y)&=\frac{1}{4 \pi} \int _{-l/2} ^{l/2} \log((x -\xi)^{2} + y^{2}) d\xi  \\&= \frac{1}{2\pi} \left( -y \arctan\left( \frac{x - l/2}{y} \right)  +y \arctan\left( \frac{x + l/2}{y} \right) -l \right) \nonumber \\
 &+ \frac{1}{4 \pi} \left(  (x+l/2) \log (y^{2} + (x+l/2)^{2})  - (x-l/2) \log (y^{2} + (x-l/2)^{2}) \right). \nonumber
\end{align}
Trak širine $l$ in potencialno polje v točki $u(x,y)$ v lastnem sistemu je prikazan na sliki  \ref{fig:slika1}.

\begin{figure}[H]
    \centering
    \begin{subfigure}[b]{0.35\textwidth}  			
        \includegraphics[width=\textwidth]{G_0_r.pdf}
    \end{subfigure}
    \caption{Greenova funkcija  panela v lastnem sistemu $G_0(0,x ,y)$.} \label{fig:slika1}
\end{figure}

\noindent Ideja tokratne naloge je, opisati neko poljubno obliko s končno mnogimi trakovi dolžin $l_i$ oz. tako imenovanimi ‘‘paneli’’. Pri tem je potrebno panele ustrezno prestaviti in rotirati, kot prikazano na sliki \ref{fig:slika2}. Potencial laboratorijskega sistema je tako potrebno poračunati v ustreznem lastnem sistemu panela.

\begin{figure}[H]
    \centering
    \begin{subfigure}[b]{\textwidth}  			
        \includegraphics[width=\textwidth]{G.pdf}
    \end{subfigure}
    \caption{Prestavljen in rotiran panel in njegov lastni sistem. Dobimo $G_{r,r'}$.} \label{fig:slika2}
\end{figure}

Za začetek si poglejmo potencialno polje enakomerno nabitega panela prikazao na sliki \ref{fig:slika3}, ki predstavlja izolator, kateri ima nanešeno konstantno gostoto naboja.

\begin{figure}[H]
    \centering
    \begin{subfigure}[b]{0.5\textwidth}  			
        \includegraphics[width=\textwidth]{ravna_zica.pdf}
    \end{subfigure}
    \caption{Izolator s konstantno gostoto naboja.} \label{fig:slika3}
\end{figure}

\noindent Rezultat je seveda pričakovan. Potrebno je opomniti, da rezultat je le približek zelo dolgega traku in nima fizikalnega pomena, saj potencial, za katerega \textit{Greenova} funkciaj za ravnino $u(r)= \frac{1}{2\pi}\log(r)$ divergira, ko $u(r \rightarrow \infty)  \rightarrow 
\infty$, kar seveda fizikalno nima iterpretacije.

Poglejmo si sedaj polje prevodnega traku. Trak razdelimo na panele. Želimo imeti konstanten potencial na traku. To dosežemo tako, da  poračunamo prispevek $j$-tega panela v središču $i$-tega, kar je element $G_{ij}$ matrike $G$. Takšna matrika je simetrična. Želimo, da velja
\begin{equation}
u_i = \sum_j G_{ij} \sigma _j = konst. \ \,
\end{equation}
kjer je $u_i$ potencial $i$-tega panela in ga postavimo na konstantno vrednost. Torej iščemo površinske gostote naboja $\sigma _i$ panelov. Z izračunano matriko $G$ je iskanje vektorja $\sigma$ trivialno. Z izračunanimi gostotami naboja $\sigma$ je potem naša rešitev v poljubni točki laboratorijskega sistema kar
\begin{equation*}
u(x,y)= \sum _i G_{ij} \sigma _i,
\end{equation*}
seveda v smislu, da so paneli ustrezno rotirani in translatirani.
Na sliki \ref{fig:slika4} je prikazan rezultat elektrostatskega polja za raven prevoden trak, razdeljen na 100 panelov. Robne točke panelov smo določili s funkcijo
\begin{equation*}
f=0.5 \cos (\pi x),
\end{equation*}
kjer je $x$ enakomerno porazdeljen med 0 in 1. Tako dobimo bolj gosto razporeditev točk na robovih traku.

\begin{figure}[H]
    \centering
    \begin{subfigure}[b]{0.5\textwidth}  			
        \includegraphics[width=\textwidth]{ravna_zica_3.pdf}
    \end{subfigure}
    \caption{Potencial prevodnega traku razdeljenega na 100 panelov.} \label{fig:slika4}
\end{figure}

\noindent Poglejmo si še razporeditev naboja po traku. Razporeditev prikazujeta sliki \ref{fig:slika5}

\begin{figure}[H]
    \centering
    \begin{subfigure}[b]{0.45\textwidth}  			
        \includegraphics[width=\textwidth]{ravna_zica_4_sigma.pdf}
        \caption{Razporeditev površinske gostote naboja $\sigma$.}
    \end{subfigure}
    \begin{subfigure}[b]{0.45\textwidth}  			
        \includegraphics[width=\textwidth]{ravna_zica_4_charge.pdf}
        \caption{Razporeditev dolžinske gostote naboja na panelu deljeno z dolžino panela $\sigma_i / l_i$.}
    \end{subfigure}
    \caption{Razporeditev naboja po traku, glede na število panelov} \label{fig:slika5}
\end{figure}
\noindent Slike \ref{fig:slika5} niso ravno zelo pregledne, vidimo pa, da se naboj nabira na vsakem od vogalov traku in z večanjem števila panelov divergira (je  $\infty$). Celoten naboj mora ostati konstanten. Konvergenco celotnega naboja na širino traku glede na število panelov prikazuje slika \ref{fig:slika6}.

\begin{figure}[H]
    \centering
    \begin{subfigure}[b]{0.5\textwidth}  			
        \includegraphics[width=\textwidth]{ravna_zica_4_naboj.pdf}
    \end{subfigure}
    \caption{$\sum _i \sigma _i  \cdot l_i$} \label{fig:slika6}
\end{figure}

Podobno analizo bi lahko napravili še za bolj poljubno postavitev elektrod s konstantnim potencialom. Primer je prikazan na sliki \ref{fig:slika7}, kjer imamo enakomerno razporejenih 400 panelov.

\begin{figure}[H]
    \centering
    \begin{subfigure}[b]{0.45\textwidth}  			
        \includegraphics[width=\textwidth]{3_paneli_2.pdf}
        \caption{Elektrostatki potencial.}
    \end{subfigure}
    \begin{subfigure}[b]{0.45\textwidth}  			
        \includegraphics[width=\textwidth]{3_paneli_2_sigma.pdf}
        \caption{Porazdelitev naboja.}
    \end{subfigure}
    \caption{Poljubna postavitev treh elektrod s konstantnim potencialom za 200 panelov.} \label{fig:slika7}
\end{figure}

\noindent Vidimo, da se naboj nabira na vogalih elektrod, kar je seveda pričakovano.

Kot zanimivost si oglejmo še kvazi kondenzator priključen na elektrodi s potencialoma 1 in -1 ter 1  in 1. Podrobne analize ne bomo naredili. Razporeditev naboja je zelo podobna razporeditvi prikazani na sliki \ref{fig:slika5}a. Rezultate prikazujeta sliki \ref{fig:slika8}.

\begin{figure}[H]
    \centering
    \begin{subfigure}[b]{0.45\textwidth}  			
        \includegraphics[width=\textwidth]{kondenzator.pdf}
    \end{subfigure}
    \begin{subfigure}[b]{0.45\textwidth}  			
        \includegraphics[width=\textwidth]{anti_kondenzator.pdf}
    \end{subfigure}
    \caption{Kondenzator in anti-kondenzator :D.} \label{fig:slika8}
\end{figure}


\section*{Obtekanje vitkih teles}

Metoda je zelo popularna v hidrodinamiki za obtekanje idealne tekočine okoli vitkih teles, kjer imamo podan vektor hitrosti v neskončnosti, robni pogoj na površini telesa pa zahteva, da je normalna hitrost enaka 0. Količine $\sigma _i$ predstavljajo gostoto izvirov, ki jih vpeljemo na meji telesa, da z njimi kompenziramo zunanji tok. Enačba (\ref{eq:enacba1}) sedaj predstavlja hitrostni  potencial enega panela v lastnem sistemu. Na enačbo (\ref{eq:enacba1}) delujemo z $\nabla _{x,y}$ in dobimo komponenti hitrosti
\begin{align}
  v_{\parallel} &=\frac{1}{4 \pi} \log\left( \frac{(x+l/2)^{2} +y^{2}}{(x-l/2)^{2} +y^{2}} \right) \label{eq:enacba3}\\
v_ {\perp}&= \frac{1}{2\pi} \left( \arctan\left( \frac{x+l}{y} \right) - \arctan\left( \frac{x-l}{y}\right) \right). \label{eq:enacba4}
\end{align}
V dobljenih enačbah je potrebno paziti, saj vzporedna komponenta divergira na vogalih, pravokotna komponenta pa ima v limiti $y \rightarrow 0$ končno vrednost
\begin{equation*}
\lim _{y \rightarrow 0} v_{\perp} = \left\{
        \begin{array}{ll}
            1/2 & \quad ;-l/2 \leq x \leq l/2 \\
            0 & \quad ;\textrm{sicer}.
        \end{array}
    \right.
\end{equation*}

Najprej si poglejmo metodo na neskončnem valju, okroglega preseka, kjer bomo za  vektor hitosti v neskončnosti vzeli
\begin{align*}
\vec{v}_{\infty}=(\sqrt{2}/2, \sqrt{2}/2).
\end{align*}
Najprej razdelimo okrogel presek na panele in izračunamo smerne vektorje $\vec{t}_i$ in normalne 
vektorje $\vec{n}_i = (-t_{y i}, t_{x i})$ panelov. Nato poračunamo normalne komponente hitrosti, ki jih povzroča tok. Profil in velikost komponent je prikazan na slikah \ref{fig:slika9}
\begin{figure}[H]
    \centering
    \begin{subfigure}[b]{0.45\textwidth}  			
        \includegraphics[width=\textwidth]{obtekanje_valj_krog_vprav1.pdf}
    \end{subfigure}
    \begin{subfigure}[b]{0.45\textwidth}  			
        \includegraphics[width=\textwidth]{obtekanje_valj_krog_vprav2.pdf}
    \end{subfigure}
    \caption{Profil normalnih komponent hitrosti panelov.} \label{fig:slika9}
\end{figure}
\noindent Poračunati moramo matriko $V$ in rešiti sistem $V \sigma =v_{\perp}$. Tokrat imamo opravka z vektorji, zato je potrebno biti previden z rotacijami. Rezultate prikazujeta spodnji sliki \ref{fig:slika10} za primer 99 enakomerno razporejenih panelov.
\begin{figure}[H]
    \centering
    \begin{subfigure}[b]{0.45\textwidth}  			
        \includegraphics[width=\textwidth]{obtekanje_valj_krog_U.pdf}
    \end{subfigure}
    \begin{subfigure}[b]{0.45\textwidth}  			
        \includegraphics[width=\textwidth]{obtekanje_valj_krog_v.pdf}
    \end{subfigure}
    \caption{Hitrostno potencialno polje levo in vektorsko hitrostno polje desno.} \label{fig:slika10}
\end{figure}
\noindent Prikažemo lahko še razporejenost izračunanih ivzirov gostote. Kot razberemo iz slike \ref{fig:slika11}, so izviri gostote raporejeni ravno obratno, kot profil normalnih hitrosti panelov.
\begin{figure}[H]
    \centering
    \begin{subfigure}[b]{0.4\textwidth}  			
        \includegraphics[width=\textwidth]{obtekanje_valj_krog_sigma.pdf}
    \end{subfigure}
    \caption{Razporejenost gostote izvirov.} \label{fig:slika11}
\end{figure}

Poglejmo si sedaj valj s presekom elipse na sto in en način. Prvi primer za vektor hitrosti
 $\vec{v}_{\infty}=(1, 0)$, rezultate prikazujejo slike \ref{fig:slika12} in \ref{fig:slika13}.

\begin{figure}[H]
    \centering
    \begin{subfigure}[b]{0.45\textwidth}  			
        \includegraphics[width=\textwidth]{obtekanje_valj_elipsa_0_vprav1.pdf}
    \end{subfigure}
    \begin{subfigure}[b]{0.45\textwidth}  			
        \includegraphics[width=\textwidth]{obtekanje_valj_elipsa_0_vprav2.pdf}
    \end{subfigure}
    \caption{Profil normalnih komponent hitrosti panelov za elipso.} \label{fig:slika12}
\end{figure}

\begin{figure}[H]
    \centering
    \begin{subfigure}[b]{0.45\textwidth}  			
        \includegraphics[width=\textwidth]{obtekanje_valj_elipsa_0_U.pdf}
    \end{subfigure}
    \begin{subfigure}[b]{0.45\textwidth}  			
        \includegraphics[width=\textwidth]{obtekanje_valj_elipsa_0_v.pdf}
    \end{subfigure}
    \caption{Hitrostno potencialno polje levo in vektorsko hitrostno polje desno za elipso.} \label{fig:slika13}
\end{figure}

\noindent Sledi primer z $\vec{v}_{\infty}=(0, 1)$ na slikah \ref{fig:slika14} in \ref{fig:slika15}.

\begin{figure}[H]
    \centering
    \begin{subfigure}[b]{0.45\textwidth}  			
        \includegraphics[width=\textwidth]{obtekanje_valj_elipsa_1_vprav1.pdf}
    \end{subfigure}
    \begin{subfigure}[b]{0.45\textwidth}  			
        \includegraphics[width=\textwidth]{obtekanje_valj_elipsa_1_vprav2.pdf}
    \end{subfigure}
    \caption{Profil normalnih komponent hitrosti panelov za elipso.} \label{fig:slika14}
\end{figure}

\begin{figure}[H]
    \centering
    \begin{subfigure}[b]{0.45\textwidth}  			
        \includegraphics[width=\textwidth]{obtekanje_valj_elipsa_1_U.pdf}
    \end{subfigure}
    \begin{subfigure}[b]{0.45\textwidth}  			
        \includegraphics[width=\textwidth]{obtekanje_valj_elipsa_1_v.pdf}
    \end{subfigure}
    \caption{Hitrostno potencialno polje levo in vektorsko hitrostno polje desno za elipso.} \label{fig:slika15}
\end{figure}

\noindent Kot zadnji primer elipse imamo $\vec{v}_{\infty}=(1/2,\sqrt{3/4)}$ na slikah \ref{fig:slika16} in \ref{fig:slika17}.

\begin{figure}[H]
    \centering
    \begin{subfigure}[b]{0.45\textwidth}  			
        \includegraphics[width=\textwidth]{obtekanje_valj_elipsa_2_vprav1.pdf}
    \end{subfigure}
    \begin{subfigure}[b]{0.45\textwidth}  			
        \includegraphics[width=\textwidth]{obtekanje_valj_elipsa_2_vprav2.pdf}
    \end{subfigure}
    \caption{Profil normalnih komponent hitrosti panelov za elipso.} \label{fig:slika16}
\end{figure}

\begin{figure}[H]
    \centering
    \begin{subfigure}[b]{0.45\textwidth}  			
        \includegraphics[width=\textwidth]{obtekanje_valj_elipsa_2_U.pdf}
    \end{subfigure}
    \begin{subfigure}[b]{0.45\textwidth}  			
        \includegraphics[width=\textwidth]{obtekanje_valj_elipsa_2_v.pdf}
    \end{subfigure}
    \caption{Hitrostno potencialno polje levo in vektorsko hitrostno polje desno za elipso.} \label{fig:slika17}
\end{figure}

Podobno kratko analizo naredimo še za profil  NACA-0020, kjer imamo rešitve v prvem primeru za $\vec{v}_{\infty}=(1,0)$ na slikah \ref{fig:slika18} in \ref{fig:slika19}, in v drugem primeru $\vec{v}_{\infty}=(-1,0)$ na slikah \ref{fig:slika20} in \ref{fig:slika21}.

\begin{figure}[H]
    \centering
    \begin{subfigure}[b]{0.45\textwidth}  			
        \includegraphics[width=\textwidth]{obtekanje_naca00_0_1.pdf}
    \end{subfigure}
    \begin{subfigure}[b]{0.45\textwidth}  			
        \includegraphics[width=\textwidth]{obtekanje_naca00_0_2.pdf}
    \end{subfigure}
    \caption{Profil normalnih komponent hitrosti panelov za NACA profil.} \label{fig:slika18}
\end{figure}

\begin{figure}[H]
    \centering
    \begin{subfigure}[b]{0.45\textwidth}  			
        \includegraphics[width=\textwidth]{obtekanje_naca00_0_U.pdf}
    \end{subfigure}
    \begin{subfigure}[b]{0.45\textwidth}  			
        \includegraphics[width=\textwidth]{obtekanje_naca00_0_v.pdf}
    \end{subfigure}
    \caption{Hitrostno potencialno polje levo in vektorsko hitrostno polje desno za NACA profil.} \label{fig:slika19}
\end{figure}

\begin{figure}[H]
    \centering
    \begin{subfigure}[b]{0.45\textwidth}  			
        \includegraphics[width=\textwidth]{obtekanje_naca00_1_1.pdf}
    \end{subfigure}
    \begin{subfigure}[b]{0.45\textwidth}  			
        \includegraphics[width=\textwidth]{obtekanje_naca00_1_2.pdf}
    \end{subfigure}
    \caption{Profil normalnih komponent hitrosti panelov za NACA profil.} \label{fig:slika20}
\end{figure}

\begin{figure}[H]
    \centering
    \begin{subfigure}[b]{0.45\textwidth}  			
        \includegraphics[width=\textwidth]{obtekanje_naca00_1_U.pdf}
    \end{subfigure}
    \begin{subfigure}[b]{0.45\textwidth}  			
        \includegraphics[width=\textwidth]{obtekanje_naca00_1_v.pdf}
    \end{subfigure}
    \caption{Hitrostno potencialno polje levo in vektorsko hitrostno polje desno za NACA profil.} \label{fig:slika21}
\end{figure}

\noindent Zanimiv rezultat dobimo, če podamo vektor $\vec{v}_{\infty}=(0,1)$. V tem primeru imamo singularno točko v vogalu profila NACA. Rezultate prikazujejo slike \ref{fig:slika22}, \ref{fig:slika23}.

\begin{figure}[H]
    \centering
    \begin{subfigure}[b]{0.45\textwidth}  			
        \includegraphics[width=\textwidth]{obtekanje_naca00_2_1.pdf}
    \end{subfigure}
    \begin{subfigure}[b]{0.45\textwidth}  			
        \includegraphics[width=\textwidth]{obtekanje_naca00_2_2.pdf}
    \end{subfigure}
    \caption{Profil normalnih komponent hitrosti panelov za NACA profil.} \label{fig:slika22}
\end{figure}

\begin{figure}[H]
    \centering
    \begin{subfigure}[b]{0.45\textwidth}  			
        \includegraphics[width=\textwidth]{obtekanje_naca00_2_U.pdf}
    \end{subfigure}
    \begin{subfigure}[b]{0.45\textwidth}  			
        \includegraphics[width=\textwidth]{obtekanje_naca00_2_v.pdf}
    \end{subfigure}
    \caption{Hitrostno potencialno polje levo in vektorsko hitrostno polje desno za NACA profil.} \label{fig:slika23}
\end{figure}
\noindent Ker se na desni sliki \ref{fig:slika23} slabše vidi hitrostno polje zaradi singularne točke v vogalu, smo zaradi lepše preglednosti postavili vse hitrosti znotraj profila postavili na $\vec{v}_{\infty}$. Rezultat z lepšim zunanjim profilom toka, je prikazan na sliki \ref{fig:slika24}

\begin{figure}[H]
    \centering
    \begin{subfigure}[b]{0.8\textwidth}  			
        \includegraphics[width=\textwidth]{obtekanje_naca00_2_v2_v.pdf}
    \end{subfigure}
    \caption{ Vektorsko hitrostno polje za NACA profil.} \label{fig:slika24}
\end{figure}

Za konec si poglejmo še NACA profil na primeru $\vec{v}_{\infty}=(-\sqrt{3/5},\sqrt{2/5})$. Rezultate prikatujejo slike \ref{fig:slika25}, \ref{fig:slika26} in zaradi lepše preglednosti hitrostnega profila \ref{fig:slika27}.

\begin{figure}[H]
    \centering
    \begin{subfigure}[b]{0.45\textwidth}  			
        \includegraphics[width=\textwidth]{obtekanje_naca00_3_1.pdf}
    \end{subfigure}
    \begin{subfigure}[b]{0.45\textwidth}  			
        \includegraphics[width=\textwidth]{obtekanje_naca00_3_2.pdf}
    \end{subfigure}
    \caption{Profil normalnih komponent hitrosti panelov za NACA profil.} \label{fig:slika25}
\end{figure}

\begin{figure}[H]
    \centering
    \begin{subfigure}[b]{0.45\textwidth}  			
        \includegraphics[width=\textwidth]{obtekanje_naca00_3_U.pdf}
    \end{subfigure}
    \begin{subfigure}[b]{0.45\textwidth}  			
        \includegraphics[width=\textwidth]{obtekanje_naca00_3_v.pdf}
    \end{subfigure}
    \caption{Hitrostno potencialno polje levo in vektorsko hitrostno polje desno za NACA profil.} \label{fig:slika26}
\end{figure}


\begin{figure}[H]
    \centering
    \begin{subfigure}[b]{0.8\textwidth}  			
        \includegraphics[width=\textwidth]{obtekanje_naca00_3_v2_v.pdf}
    \end{subfigure}
    \caption{ Vektorsko hitrostno polje za NACA profil.} \label{fig:slika27}
\end{figure}

\end{document}
