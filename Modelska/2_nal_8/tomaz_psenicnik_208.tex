\documentclass[12pt,a4paper]{article}

\usepackage[utf8x]{inputenc}   % omogoča uporabo slovenskih črk kodiranih v formatu UTF-8
\usepackage[slovene]{babel}    % naloži, med drugim, slovenske delilne vzorce

\usepackage{subcaption}
\usepackage[hyphens]{url}
\usepackage{hyperref}


\usepackage[pdftex]{graphicx}
\usepackage{wrapfig}

\usepackage{amsmath}
%\renewcommand{\vec}[1]{\boldsymbol{#1}} %naredi vector kot bold zapis

\usepackage{float}

\usepackage{amssymb}

%\documentclass[a4paper, 12pt]{article}
%\usepackage[slovene]{babel}
%\usepackage[latin2]{inputenc}
%\usepackage[T1]{fontenc}
%\usepackage{makeidx}%za stvarno kazalo
%\makeindex%naredi stvarno kazalo
%\usepackage{tikz}% paket za kroge

\title{\textbf{Modelska analiza 2} \\ 8. naloga - Metoda končnih elementov: Lastne rešitve \\}
	\author{Študent: Pšeničnik Tomaž}
	
	


	
\begin{document}

\pagenumbering{gobble}

	\begin{figure} [h]
  \centering
  \includegraphics[width=12 cm]{logo_fmf.png}
  \maketitle
\end{figure}
	
	
	
	\newpage
	\pagenumbering{arabic}
	
	
	
\section*{Lastne rešitve FEM - Teorija}

Z metodo končnih elementov (FEM) želimo poračunati lastne nihajne načine polkrožne opne. Problem je zelo podoben reševanju \textit{Poissonove} enačbe v prejšnji nalogi, le da tokrat rešujemo s pomočjo lastnih vrednosti. Enačba se tokrat glasi 
\begin{equation} \label{eq:enacba1}
\nabla ^{2} u =-k^{2} u,
\end{equation}
kjer je $k$ lastna vrednost sistema.

Podobno kot dobimo \textit{Poissonovo} enačbo z zahtevo po ekstremu funcionala $\mathcal{S}(u) = \frac{1}{2} \left\langle \nabla u , \nabla u \right\rangle - \left\langle f,u \right\rangle$, dobimo tudi enačbo (\ref{eq:enacba1}) z zahtevo po ekstremu funkcionala
\begin{equation} \label{eq:enacba2}
\mathcal{S}(u) = \frac{1}{2} \left\langle \nabla u , \nabla u \right\rangle - \frac{1}{2} k^{2}\left\langle u,u \right\rangle,
\end{equation}
kjer je skalarni produkt definiran  na $L^{2}$ prostoru kot
\begin{align*}
\left\langle a ,b \right\rangle= \int a \cdot b  \ \ dxdy.
\end{align*}
Če zapišemo funkcional (\ref{eq:enacba2}) kot $\mathcal{S}(u)=\int \mathcal{L}(\nabla u,u) dx dy$, dobimo z \textit{Euler-Lagrangevo} enačbo:
\begin{align*}
\sum _i \frac{\partial}{ \partial x_i} \frac{ \partial \mathcal{L}}{ \partial u_{x{_i}}} - \frac{ \partial \mathcal{L}}{\partial u}=0
\end{align*}
ravno enačbo (\ref{eq:enacba1}). Toda naš cilj ni reševanje E-L enačb, temveč varrirati funckional (\ref{eq:enacba2}) po baznih funkcijah, kar je enako iskanju ekstrema funkcionala. 

Kot v prejšnji nalogi, tudi tokrat razvijemo iskano funkcijo $u$ po baznih funkcijah
\begin{align*}
u=\sum_{i=1} ^{N} c_i \phi _i 
\end{align*}
in vstavimo v funkcional (\ref{eq:enacba2}). Dobimo
\begin{equation} \label{eq:enacba3}
\mathcal{S}=\frac{1}{2} \sum _{ij} \left\langle  c_i \phi _i , c_j \phi_j   \right\rangle - \frac{1}{2} k^{2} \sum _{ij} \left\langle  c_i \phi _i , c_j \phi_j   \right\rangle.
\end{equation}
Kot omenjeno, iščemo ekstrem funkcionala (\ref{eq:enacba3}) po baznih funkcijah, kar prepoznamo kot sistem linearnih enačb, ki ga zapišemo:
\begin{equation}
\frac{ \partial \mathcal{S}}{\partial c_i}= 0 \rightarrow \sum _{ij} (A_{ij} - k^{2} B_{ij})c_j = 0  \qquad  \forall i.
\end{equation}
Za bazne funkcije bomo vzeli iste funkcije kot v prejšnji nalogi.
Izračun lokalne matrike $A^{k}_{ij}$ tako poznamo že od prejšnje naloge in je enak:
\begin{equation*}
A^{k}_{ij}=\left\langle   \nabla \phi _i , \nabla \phi_j   \right\rangle =\frac{1}{2|J_k|} (y_{i+1} - y_{i+2} , \ \ x_{i+2} - x_{i+1})
\begin{pmatrix}
 y_{j+1} - y_{j+2} \\ x_{j+1} - x_{j+1}
\end{pmatrix}.
\end{equation*}
Za lokalno $B^{k}_{ij}$ matriko pa velja
\begin{equation*}
\left\langle  \phi _i , \phi_j   \right\rangle = \int _{\bigtriangleup_{k}} \phi _i \phi _j dx dy =\int _{\bigtriangleup ^{*}} \chi _i \chi _j |J_k| d\xi d\eta,
\end{equation*}
kjer je $\phi_i$ bazna funkcija trikotnika $\bigtriangleup _{k}$ podanega območja in $\chi_i$ bazna funkcija referenčnega trikotnika $\bigtriangleup_{*}$, $|J_k|=2\bigtriangleup_{k}$ pa Jacobijeva determinanta, ki je konstantna in enaka dvakratni ploščini $\bigtriangleup_{k}$. Tako dobimo za integral po baznih funkcijah referenčnega trikotnika
\begin{equation*}
B_{ii} ^{k}= \int _{\bigtriangleup _{*}} \chi _i \chi _i |J_k|d\xi d\eta = \frac{|J_k|}{12} \qquad \textrm{in} \qquad B_{i\neq j}^{k} =  \int _{\bigtriangleup _{*}} \chi _i \chi _j |J_k|d\xi d\eta = \frac{|J_k|}{24}
\end{equation*}
oziroma, za območje $\bigtriangleup_{k}$
\begin{equation}
B_{ii}^{k} = \frac{1}{6} \bigtriangleup_{k} \qquad \textrm{in} \qquad B_{i\neq j}^{k}=\frac{1}{12} \bigtriangleup_{k}.
\end{equation}
Potrebno je sestaviti globalni matriki $A$ in $B$ ter tako poračunati sistem 
\begin{align*}
A \vec{c}= k^{2} B \vec{c},
\end{align*}
kjer je rešitev
\begin{align*}
u(x,y)=\sum_i c_i \phi_i.
\end{align*}

\section*{Lastni nihajni načini kvadratne opne}

Kljub temu, da je naša naloga poiskati lastne nihanje načine opne, si bomo za začetek pogledali lastne nihajne načine kvadratne opne. Na slikah (\ref{fig:slika1}) sta prikazani matriki $A$ in $B$, kjer upoštevamo le notranje elemente triangulacije, prikazane na sliki \ref{fig:slika2}. Matrika $A$ je identična matriki diskretnih diferencialov, ki smo jo srečali v 5. in 6. nalogi. To je slučaj zaradi izbire porazdelitve točk. Obliko matrike $B$ vidimo prvič. Obe matriki sta redki (\textit{sparse}) in simetrični, zato lahko za iskanje lastnih vrednosti uporabimo knjižico \texttt{scipy.sparse.linalg.eigsh}.

\begin{figure}[H]
    \centering
    \begin{subfigure}[b]{0.45\textwidth}  			
        \includegraphics[width=\textwidth]{kvadrat_A.pdf}
        \caption{Globalna matrika $A$.}
    \end{subfigure}
    \begin{subfigure}[b]{0.45\textwidth}  			
        \includegraphics[width=\textwidth]{kvadrat_B.pdf}
        \caption{Globalna matrika $B$.}
    \end{subfigure}
    \caption{Globalni matriki za kvadratno opno.} \label{fig:slika1}
\end{figure}

\begin{figure}[H]
    \centering
    \begin{subfigure}[b]{0.45\textwidth}  			
        \includegraphics[width=\textwidth]{kvadrat_tri.pdf}
    \end{subfigure}
    \caption{Triangulacija za kvadratno opno za 10x10 točk.} \label{fig:slika2}
\end{figure}

\noindent Lastne nihajne načine smo poiskali za 50x50 točk. Lastne vrednosti za prve štiri lastne nihajne načine so prikazane v tabeli \ref{table:tabela1}. Na slikah \ref{fig:slika3} so nato prikazani lastni nihajni načini pridobleni s FEM metodo v jeziku \textit{Python}.

\begin{table}[H]
\begin{center}
\begin{tabular}{|c|c|c|c|} 
\hline
n & i j                                               & $\pi ^{2} (i^{2} + j^{2})$ & $k^{2}$                                                   \\ \hline
1 & 1 1                                               & 19.7392                    & 19.7584                                                   \\ \hline
2 & \begin{tabular}[c]{@{}l@{}}1 2\\ 2 1\end{tabular} & 49.3480                    & \begin{tabular}[c]{@{}l@{}}49.4512\\ 49.4516\end{tabular} \\ \hline
3 & 2 2                                               & 78.9568                    & 79.2920                                                   \\ \hline
4 & \begin{tabular}[c]{@{}l@{}}3 2\\ 2 3\end{tabular} & 98.6960                    & \begin{tabular}[c]{@{}l@{}}99.0636\\ 99.0680\end{tabular} \\ \hline
\end{tabular}
\caption{Lastne vrednosti za prve štiri lastne nihajne načine.}
\label{table:tabela1}
\end{center}
\end{table}

\begin{figure}[H]
    \centering
    \begin{subfigure}[b]{0.45\textwidth}  			
        \includegraphics[width=\textwidth]{kvadrat__lastne_1.pdf}
        \caption{Prvi nihajni način.}
    \end{subfigure}
    \begin{subfigure}[b]{0.45\textwidth}  			
        \includegraphics[width=\textwidth]{kvadrat__lastne_4.pdf}
        \caption{Tretji nihajni način.}
    \end{subfigure}
    
    \begin{subfigure}[b]{0.45\textwidth}  			
        \includegraphics[width=\textwidth]{kvadrat__lastne_2.pdf}
        \caption{Drugi nihajni način.}
    \end{subfigure}
    \begin{subfigure}[b]{0.45\textwidth}  			
        \includegraphics[width=\textwidth]{kvadrat__lastne_3.pdf}
        \caption{Drugi nihajni način.}
    \end{subfigure}
    
    \begin{subfigure}[b]{0.45\textwidth}  			
        \includegraphics[width=\textwidth]{kvadrat__lastne_5.pdf}
        \caption{Četrti nihajni način.}
    \end{subfigure}
    \begin{subfigure}[b]{0.45\textwidth}  			
        \includegraphics[width=\textwidth]{kvadrat__lastne_6.pdf}
        \caption{Četrti nihajni način.}
    \end{subfigure}
    \caption{Lastni nihajni načini kvadratne opne izračunani v jeziku \textit{Python}.} \label{fig:slika3}
\end{figure}

\noindent Poračunali pa smo še lastne nihajne načine s pomočjo programa \textit{Matlab}, kateri ima vgrajeno orodje za reševanje parcialnih diferencialnih enačb. Prikličemo ga z ukazom \texttt{pdetool}. Rešitve so prikazane na slikah \ref{fig:slika4}.

\begin{figure}[H]
    \centering
    \begin{subfigure}[b]{0.45\textwidth}  			
        \includegraphics[width=\textwidth]{kvadrat_lastna1.png}
        \caption{Prvi nihajni način.}
    \end{subfigure}
    \begin{subfigure}[b]{0.45\textwidth}  			
        \includegraphics[width=\textwidth]{kvadrat_lastna4.png}
        \caption{Tretji nihajni način.}
    \end{subfigure}
    
    \begin{subfigure}[b]{0.45\textwidth}  			
        \includegraphics[width=\textwidth]{kvadrat_lastna2.png}
        \caption{Drugi nihajni način.}
    \end{subfigure}
    \begin{subfigure}[b]{0.45\textwidth}  			
        \includegraphics[width=\textwidth]{kvadrat_lastna3.png}
        \caption{Drugi nihajni način.}
    \end{subfigure}
    
    \begin{subfigure}[b]{0.45\textwidth}  			
        \includegraphics[width=\textwidth]{kvadrat_lastna5.png}
        \caption{Četrti nihajni način.}
    \end{subfigure}
    \begin{subfigure}[b]{0.45\textwidth}  			
        \includegraphics[width=\textwidth]{kvadrat_lastna6.png}
        \caption{Četrti nihajni način.}
    \end{subfigure}
    \caption{Lastni nihajni načini kvadratne opne izračunani v programu\textit{Matlab}.} \label{fig:slika4}
\end{figure}




\section*{Lastni nihajni načini polkrožne opne}

Reševanje je identično kot za primer kvadratne opne, le da imamo tokrat drugačno triangulacijo območja. Triangulacija je prikazana na sliki \ref{fig:slika5}, kjer imamo 406 notranjih točk in 91 robnih točk.

\begin{figure}[H]
    \centering
    \begin{subfigure}[b]{0.45\textwidth}  			
        \includegraphics[width=\textwidth]{polkrog_tri.pdf}
    \end{subfigure}
    \caption{Triangulacija za polkrožno opno za 497 točk.} \label{fig:slika5}
\end{figure}

\noindent Ponovno si oglejmo globalni matriki $A$ in $B$, prikazani na slikah \ref{fig:slika6}. Matriki sta še vedno simetrični in redki (\textit{sparse}), kar ni presenetljivo. Vidimo tudi, da zaradi drugačne porazdelitve točk ne dobimo več matrike $A$ z elementi 4 in -1, temveč se razlikujejo. Izvendiagonalni elementi se širijo narazen. Vzrok tiči v izbiri točk za triangulacijo, namreč na vsakem polkrogu od $r=0$ do $r=R$ imamo več točk, zato pride tudi do razširanjanja izvendiagonalnih elementov. Za izračun lastnih nihajnih načinov ponovno uporabimo knjižico \texttt{scipy.sparse.linalg.eigsh}.

\begin{figure}[H]
    \centering
    \begin{subfigure}[b]{0.45\textwidth}  			
        \includegraphics[width=\textwidth]{polkrog_A.pdf}
        \caption{Globalna matrika $A$.}
    \end{subfigure}
    \begin{subfigure}[b]{0.45\textwidth}  			
        \includegraphics[width=\textwidth]{polkrog_B.pdf}
        \caption{Globalna matrika $B$.}
    \end{subfigure}
    \caption{Globalni matriki za polkrožno opno.} \label{fig:slika6}
\end{figure}

\noindent Lastne nihajne načine smo poiskali za 406 notranjih točk. Lastne vrednosti za prvih šest lastnih nihajnih načinov polkrožne opne, so prikazane v tabeli \ref{table:tabela2}. Na slikah \ref{fig:slika7} so nato prikazani lastni nihajni načini pridobleni s FEM metodo v jeziku \textit{Python}.



\begin{figure}[H]
    \centering
    \begin{subfigure}[b]{0.45\textwidth}  			
        \includegraphics[width=\textwidth]{polkrog_lastne_1.pdf}
        \caption{Prvi nihajni način $i=1$, $j=1$.}
    \end{subfigure}
    \begin{subfigure}[b]{0.45\textwidth}  			
        \includegraphics[width=\textwidth]{polkrog_lastne_2.pdf}
        \caption{Drugi nihajni način $i=2$, $j=1$.}
    \end{subfigure}
    
    \begin{subfigure}[b]{0.45\textwidth}  			
        \includegraphics[width=\textwidth]{polkrog_lastne_3.pdf}
        \caption{Tretji nihajni način $i=3$, $j=1$.}
    \end{subfigure}
    \begin{subfigure}[b]{0.45\textwidth}  			
        \includegraphics[width=\textwidth]{polkrog_lastne_4.pdf}
        \caption{Četrti nihajni način $i=1$, $j=2$.}
    \end{subfigure}
    
    \begin{subfigure}[b]{0.45\textwidth}  			
        \includegraphics[width=\textwidth]{polkrog_lastne_5.pdf}
        \caption{Peti nihajni način $i=4$, $j=1$.}
    \end{subfigure}
    \begin{subfigure}[b]{0.45\textwidth}  			
        \includegraphics[width=\textwidth]{polkrog_lastne_6.pdf}
        \caption{Šesti nihajni način $i=2$, $j=2$.}
    \end{subfigure}
    \caption{Lastni nihajni načini polkrožne opne izračunani v jeziku \textit{Python}.} \label{fig:slika7}
\end{figure}

\begin{table}[H]
\begin{center}
\begin{tabular}{|c|c|c|c|c|}
\hline
n & i j & $\xi ^{j} _i /0.5$ & $k^{2}$  & $\xi ^{j} _i$ \\ \hline
1 & 1 1 & 58,7278                    & 59,0190  & 3,8317        \\ \hline
2 & 2 1 & 105,4985                   & 106,3977 & 5,1356        \\ \hline
3 & 3 1 & 162,8259                   & 164,8956 & 6,3802        \\ \hline
4 & 1 2 & 196,8738                   & 200,3545 & 7,0156        \\ \hline
5 & 4 1 & 230,3318                   & 234,6404 & 7,5883        \\ \hline
6 & 2 2 & 283,4000                   & 290,4187 & 8,4172        \\ \hline
\end{tabular}
\caption{Lastne vrednosti za prvih 6 lastnih nihajnih načinov polkrožne opne dobljenih v jeziku \textit{Python}.}
\label{table:tabela2}
\end{center}
\end{table}



\section*{Lastni nihajni načini danega lika}

Iskanje lastnih nihajnih načinov je mnogo lažje v orodju \textit{Matlab} v katerem hitreje podamo območje iskanja, triangulacijo pa izvede program sam. Tako smo dobili rezultate, prikazane na slikah \ref{fig:slika8} in \ref{fig:slika9}.

\begin{figure}[H]
    \centering
    \begin{subfigure}[b]{0.4\textwidth}  			
        \includegraphics[width=\textwidth]{lik_lastna1.png}
        \caption{Prvi nihajni način}
    \end{subfigure}
        \begin{subfigure}[b]{0.4\textwidth}  			
        \includegraphics[width=\textwidth]{lik_lastna2.png}
        \caption{Drugi nihajni način}
    \end{subfigure}
    
        \begin{subfigure}[b]{0.4\textwidth}  			
        \includegraphics[width=\textwidth]{lik_lastna3.png}
        \caption{Tretji nihajni način}
    \end{subfigure}
        \begin{subfigure}[b]{0.4\textwidth}  			
        \includegraphics[width=\textwidth]{lik_lastna4.png}
        \caption{Četrti nihajni način}
    \end{subfigure}
    \caption{Prvi štirje lastni nihajni načini danega lika} \label{fig:slika8}
\end{figure}

\begin{figure}[H]
    \centering
    \begin{subfigure}[b]{0.4\textwidth}  			
        \includegraphics[width=\textwidth]{lik_lastna5.png}
        \caption{Peti nihajni način}
    \end{subfigure}
        \begin{subfigure}[b]{0.4\textwidth}  			
        \includegraphics[width=\textwidth]{lik_lastna6.png}
        \caption{Šesti nihajni način}
    \end{subfigure}
    
        \begin{subfigure}[b]{0.4\textwidth}  			
        \includegraphics[width=\textwidth]{lik_lastna7.png}
        \caption{Sedmi nihajni način}
    \end{subfigure}
        \begin{subfigure}[b]{0.4\textwidth}  			
        \includegraphics[width=\textwidth]{lik_lastna8.png}
        \caption{Osmi nihajni način}
    \end{subfigure}
    \caption{Peti, šesti, sedmi in osmi lastni nihajni načini danega lika.} \label{fig:slika9}
\end{figure}

\section*{Nastavek Galerkinova}

Poskusimo oceniti lastne nihajne vrednosti za polkrožno opno z \textit{Galerkinovim} nastavkom
\begin{equation} \label{eq:enacba6}
\phi_i \rightarrow \phi _{mn}(r,\varphi) = r^{m+n} (1-r) \sin(m \varphi),
\end{equation}
ki je bazna funkcija napeta čez celoten naš polkrog, pri čemer je
\begin{equation*}
1 \leq m \leq m_{max} \qquad  \textrm{in} \qquad 0 \leq n \leq n_{max}.
\end{equation*}
Za začetek si poglejmo nekaj baznih funkcij, prikazanih na slikah \ref{fig:slika10}. Določene bazne funkcije spominjajo na lastne nihajne načine. Ideja je tako iz nekaj izbranih baznih funckij skonstruirati lastne nihajne načine in poiskati lastne vrednosti. Skonstruirati je potrebno matriko $A$ in matriko $B$ ter rešiti sistem
\begin{equation*}
A \vec{c} = k^{2} B \vec{c}.
\end{equation*}
Tako dobimo rešitev
\begin{align*}
u(x,y)= \sum _i c_i \phi _i.
\end{align*}

\begin{figure}[H]
    \centering
    \begin{subfigure}[b]{0.3\textwidth}  			
        \includegraphics[width=\textwidth]{polkrog_galerkin_baznefun_10.pdf}
        \caption{$m=1$, $n=0$.}
    \end{subfigure}
    \begin{subfigure}[b]{0.3\textwidth}  			
        \includegraphics[width=\textwidth]{polkrog_galerkin_baznefun_11.pdf}
        \caption{$m=1$, $n=1$.}
    \end{subfigure}
    \begin{subfigure}[b]{0.3\textwidth}  			
        \includegraphics[width=\textwidth]{polkrog_galerkin_baznefun_12.pdf}
        \caption{$m=1$, $n=2$.}
    \end{subfigure}
    
    \begin{subfigure}[b]{0.3\textwidth}  			
        \includegraphics[width=\textwidth]{polkrog_galerkin_baznefun_20.pdf}
        \caption{$m=2$, $n=0$.}
    \end{subfigure}
    \begin{subfigure}[b]{0.3\textwidth}  			
        \includegraphics[width=\textwidth]{polkrog_galerkin_baznefun_21.pdf}
        \caption{$m=2$, $n=1$.}
    \end{subfigure}
    \begin{subfigure}[b]{0.3\textwidth}  			
        \includegraphics[width=\textwidth]{polkrog_galerkin_baznefun_22.pdf}
        \caption{$m=2$, $n=2$.}
    \end{subfigure}
    
    \begin{subfigure}[b]{0.3\textwidth}  			
        \includegraphics[width=\textwidth]{polkrog_galerkin_baznefun_30.pdf}
        \caption{$m=3$, $n=0$.}
    \end{subfigure}
    \begin{subfigure}[b]{0.3\textwidth}  			
        \includegraphics[width=\textwidth]{polkrog_galerkin_baznefun_31.pdf}
        \caption{$m=3$, $n=1$.}
    \end{subfigure}
    \begin{subfigure}[b]{0.3\textwidth}  			
        \includegraphics[width=\textwidth]{polkrog_galerkin_baznefun_32.pdf}
        \caption{$m=3$, $n=2$.}
    \end{subfigure}
    \caption{Nekaj prvih baznih funckij.} \label{fig:slika10}
\end{figure}

Konstrukcija matrik je zelo podobna prejšnji metodi FEM:
\begin{align*}
A_{ij} \rightarrow A_{(mn)(kl)}= \left\langle \nabla \phi_{mn} , \nabla \phi_{kl} \right\rangle \\
B_{ij} \rightarrow B_{(mn)(kl)}= \left\langle  \phi_{mn} ,  \phi_{kl} \right\rangle,
\end{align*}
tako dobimo bločni matriki. Tokrat moramo je skalarni produkt $\left\langle  a ,  b \right\rangle$ definiran drugače, in sicer:
\begin{align*}
\left\langle  a ,  b \right\rangle = \int _0 ^{1} r dr \int_0 ^{\pi} a \cdot b \ \ d \varphi .
\end{align*}
Funkcije $\sin (m \varphi)$ so med seboj ortogonalne in nam olajšajo delo, saj je potrebno poračunati le diagonalne elemente bločnih matrik. Matriki $A$ in $B$ sta torej enaki:
\begin{align}
A_{(mn),(ml)} & =\left\langle \nabla \phi_{mn} , \nabla \phi_{ml} \right\rangle \nonumber \\ & = \frac{\pi}{2} \left( \frac{nl}{2m +n +l} - \frac{2nl + n +l}{2m +n +l +1} + \frac{(n+1)(l+1)}{2m +n +l+2}\right),
\end{align}
\begin{align}
 B_{(mn),(ml)} & =\left\langle \phi_{mn} , \phi_{ml} \right\rangle \nonumber \\ & = \frac{\pi}{2} \left( \frac{1}{2m +n +l +2} - \frac{2}{2m +n +l +3} + \frac{1}{2m +n +l+4}\right).
\end{align}
Če si pogledamo izračunani matriki na sliki \ref{fig:slika11}, opazimo, da sta simetrični. Zato lahko za pridobitev lastnih vrednosti in vektorjev uporabimo knjižico \texttt{scipy.linalg.eigh}.

\begin{figure}[H]
    \centering
    \begin{subfigure}[b]{0.35\textwidth}  			
        \includegraphics[width=\textwidth]{polkrog_galerkin_lastne_A.pdf}
    \end{subfigure}
    \begin{subfigure}[b]{0.35\textwidth}  			
        \includegraphics[width=\textwidth]{polkrog_galerkin_lastne_B.pdf}
    \end{subfigure}
    \caption{Matrika $A$ levo in matrika $B$ desno za $m=3$ in $n=4$.} \label{fig:slika11}
\end{figure}

\noindent Poračunali smo lastne vrednosti za prvih 13 lastnih vrednosti. Primerjava z analitičnimi je prikazana v tabeli \ref{table:tabela3}.

\begin{table}[H]
\begin{center}
\begin{tabular}{|c|c|c|c|c|}
\hline
n & i j & $(\xi ^{j}_i)^{2}$ & $k^{2}$  & $\xi ^{j} _i$ \\ \hline
1 & 1 1 & 14,6819                    & 14,6820  & 3,8317        \\ \hline
2 & 2 1 & 26,3746                   & 26,3746 & 5,1356        \\ \hline
3 & 3 1 &   40,7065                 & 40,7065 & 6,3802        \\ \hline
4 & 1 2 & 49,2185                   & 49,2185 & 7,0156        \\ \hline
5 & 4 1 & 57,5829                  & 57,5829 & 7,5883        \\ \hline
6 & 2 2 & 70,8500                   & 70,8500 & 8,4172        \\ \hline
7 & 1 5 &  76,9389                  & 76,9389 & 8,7715        \\ \hline
8 & 3 2 & 95,2776                  & 95,2776 & 9,7610       \\ \hline
9 & 6 1 & 98,7263                   & 98,7263 & 9,9361        \\ \hline
10 & 1 3 & 103,4995                   & 103,5073 & 10,1735        \\ \hline
11 & 4 2 & 122,4278                  & 122,4279 & 11,0647        \\ \hline
12 & 7 1 & 122,9076                   & 122,9076 & 11,0864        \\ \hline
13 & 2 3 & 135,0207                  & 135,0545 & 11,6198         \\ \hline
14 & 8 1 & 149,4529                   & 149,4529 & 12,2251        \\ \hline
15 & 5 2 & 152,2412                  & 152,2414 & 12,3386        \\ \hline
\end{tabular}
\caption{Nekaj lastnih vrednosti za $m=10$ in $n=7$.}
\label{table:tabela3}
\end{center}
\end{table}

\noindent Poračunane lastne vrednosti se izjemno ujemajo z analitičnimi. Rezultat je veliko boljši kot pri metodi s trikotniki, tudi izračun lastnih vrednosti in lastnih vektorjev je časovno zelo kratek. Pokazali smo, da lahko z ustreznimi baznimi funkcijami precej hitreje in veliko natančneje dobimo rezultate.

Lastne nihajne načine lahko skonstruiramo iz dobljenih lastnih vektorjev. Poglejmo si le nekaj višjih lastnih nihajnih načinov, prikazanih na slikah \ref{fig:slika12}.

\begin{figure}[H]
    \centering
    \begin{subfigure}[b]{0.3\textwidth}  			
        \includegraphics[width=\textwidth]{polkrog_galerkin_lastne_6.pdf}
        \caption{7 nihajni način.}
    \end{subfigure}
    \begin{subfigure}[b]{0.3\textwidth}  			
        \includegraphics[width=\textwidth]{polkrog_galerkin_lastne_7.pdf}
        \caption{8 nihajni način.}
    \end{subfigure}
        \begin{subfigure}[b]{0.3\textwidth}  			
        \includegraphics[width=\textwidth]{polkrog_galerkin_lastne_8.pdf}
        \caption{9 nihajni način.}
    \end{subfigure}
    
        \begin{subfigure}[b]{0.3\textwidth}  			
        \includegraphics[width=\textwidth]{polkrog_galerkin_lastne_9.pdf}
        \caption{10 nihajni način.}
    \end{subfigure}
        \begin{subfigure}[b]{0.3\textwidth}  			
        \includegraphics[width=\textwidth]{polkrog_galerkin_lastne_10.pdf}
        \caption{11 nihajni način.}
    \end{subfigure}
    \begin{subfigure}[b]{0.3\textwidth}  			
        \includegraphics[width=\textwidth]{polkrog_galerkin_lastne_11.pdf}
        \caption{12 nihajni način.}
    \end{subfigure}
    
    \begin{subfigure}[b]{0.3\textwidth}  			
        \includegraphics[width=\textwidth]{polkrog_galerkin_lastne_12.pdf}
        \caption{13 nihajni način.}
    \end{subfigure}
        \begin{subfigure}[b]{0.3\textwidth}  			
        \includegraphics[width=\textwidth]{polkrog_galerkin_lastne_13.pdf}
        \caption{14 nihajni način.}
    \end{subfigure}
    \begin{subfigure}[b]{0.3\textwidth}  			
        \includegraphics[width=\textwidth]{polkrog_galerkin_lastne_14.pdf}
        \caption{15 nihajni način.}
    \end{subfigure}
    \caption{Nekaj lastnih nihajnih načinov.} \label{fig:slika12}
\end{figure}

\end{document}
