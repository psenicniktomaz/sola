\documentclass[12pt,a4paper]{article}

\usepackage[utf8x]{inputenc}   % omogoča uporabo slovenskih črk kodiranih v formatu UTF-8
\usepackage[slovene]{babel}    % naloži, med drugim, slovenske delilne vzorce

\usepackage{subcaption}
\usepackage[hyphens]{url}
\usepackage{hyperref}


\usepackage[pdftex]{graphicx}
\usepackage{wrapfig}

\usepackage{amsmath}
%\renewcommand{\vec}[1]{\boldsymbol{#1}} %naredi vector kot bold zapis

\usepackage{float}

\usepackage{amssymb}

%\documentclass[a4paper, 12pt]{article}
%\usepackage[slovene]{babel}
%\usepackage[latin2]{inputenc}
%\usepackage[T1]{fontenc}
%\usepackage{makeidx}%za stvarno kazalo
%\makeindex%naredi stvarno kazalo
%\usepackage{tikz}% paket za kroge

\title{\textbf{Modelska analiza 2} \\ 5. naloga - Parcialne diferencialne enačbe: robni problemi in relaksacija \\}
	\author{Študent: Pšeničnik Tomaž}
	
	


	
\begin{document}

\pagenumbering{gobble}

	\begin{figure} [h]
  \centering
  \includegraphics[width=12 cm]{logo_fmf.png}
  \maketitle
\end{figure}
	
	
	
	\newpage
	\pagenumbering{arabic}
	
	
	
\section*{Pretok viskozne tekočine po cevi}

Z metodo pospešene relaksacije (SOR - Successive Over-Relaxation) želimo določiti Poiseuillov koeficient za pretok viskozne tekočine po cevi z obliko preseka, podano na sliki \ref{fig:slika1}.

\begin{figure}[H]
\begin{center}
\includegraphics[width=0.3\textwidth]{lik.pdf}
\caption{Cev z danim prerezom} \label{fig:slika1}
\end{center}
\end{figure} 

\noindent Najprej si poglejmo, kaj je Poiseuillov koeficient za pretok nestisljive viskozne tekočine po cevi z nekim presekom. Po predpostavki o stacionarnem toku in gabanju le v eni smeri, recimo v smeri $z$, lahko našo enačbo gibanja zapišemo kot:
\begin{equation} \label{eq:enacba1}
-\eta \nabla ^{2} u = \frac{dp}{dz} \approx \frac{\Delta p}{ \Delta z},
\end{equation}
kjer je $\eta$ koeficient viskoznosti, $u$ hitrost tekočine v točki $(x,y)$ v smeri $z$ in $dp/dz$ gradient tlaka v smeri toka tekočine ($z$). po preureditvi enačbe (\ref{eq:enacba1}) dobimo Poissonovo enačbo
\begin{equation} \label{eq:enacba2}
-\nabla ^{2} u = q, \qquad q = \frac{1}{\eta}\frac{\Delta p}{\Delta z},
\end{equation} 
z zahtevo, da je na robu cevi $\partial R$ hitrost tekočine enaka nič $u=0$. Poiseuillov koeficient je za poljuben presek definiran kot razmerje volumskega pretoka za okroglo cev
\begin{equation}
C= \frac{\Phi_V}{\Phi_V ^{0}}, \qquad \Phi_V ^{0}= \frac{q S^{2}}{8 \pi},
\end{equation}
kjer je volumski pretok definiran kot
\begin{equation}
\Phi _V = \int _S u dS = C\frac{S^{2}q}{8 \pi}.
\end{equation}

Metoda pospešene relaksacije je iterativna metoda, pri kateri uporabimo Jacobijev ali Gauss-Seidlov postopek tako, da skalira iteracijski faktor $0 <\omega<2$:
\begin{equation*}
u_{j,k} ^{n+1} = u_{j,k} ^{n} + \omega \left( \tilde{u}_{j,k} ^{n} - u_{j,k} ^{n}  \right),
\end{equation*}
kjer je $\tilde{u_{j,k}}^{n}$ vmesni izračun Jacobijeve ali Gauss-Seidlove iteracije. Uporabili smo zapis 
\begin{equation*}
u_{j,k}=u(x_j,y_k).
\end{equation*}
Za naslednji korak moramo upoštevati še diferenčni zapis enačbe (\ref{eq:enacba2}). V primeru enako velikih korakov $h$ v vseh smereh, se ta glasi:
\begin{equation} \label{eq:enacba5}
\nabla ^{2} u= \frac{\partial ^{2} u }{\partial x^{2}} + \frac{\partial ^{2} u }{\partial x^{y}} = \frac{u_{j-1,k}+ u_{j+1,k} + u _{j,k-1} + u_{j,k+1} - 4u_{j,k} }{h^{2}}=Q_{j,k}.
\end{equation}
Iteraciji sta sledeči
\begin{itemize}
\item Jacobijeva: \\
\begin{equation} \label{eq:enacba6}
\tilde{u}_{j,k} ^{n}= u_{j,k} ^{n+1}= \frac{1}{d}\left[Q_{j,k} + \frac{u_{j+1,k}^{n} + u_{j-1,k}^{n}}{(\Delta x)^{2}} + \frac{u_{j,k+1}^{n} + u_{j,k-1}^{n}}{(\Delta y)^{2}} \right],
\end{equation}
\item Gauss-Seidel:\\
\begin{equation}
\tilde{u}_{j,k} ^{n}= u_{j,k} ^{n+1}= \frac{1}{d}\left[Q_{j,k} + \frac{u_{j+1,k}^{n} + u_{j-1,k}^{n+1}}{(\Delta x)^{2}} + \frac{u_{j,k+1}^{n} + u_{j,k-1}^{n+1}}{(\Delta y)^{2}} \right],
\end{equation}
\end{itemize}
pri čemer je
\begin{align*}
d= 2 \left( \frac{1}{(\Delta x)^{2}} +\frac{1}{(\Delta y)^{2}} \right).
\end{align*}
Optimalna izbira faktorja pospešitve $\omega$ je odvisna od spektralnega radija iteracijske matrike  $\rho_{jacobi}$, ki je 
\begin{equation}
 \rho_{jacobi}= \frac{2}{d}\left(\frac{1}{(\Delta x)^{2}}\cos \frac{\pi}{N_x}  +\frac{1}{(\Delta y)^{2}}\cos \frac{\pi}{N_y}     \right) \approx 1- \frac{\pi ^{2}}{2N^{2}},
\end{equation}
Pri čemer je v našem primeru $N_x = N_y >>1$. Optimalna izbira faktorja $\omega$ za poln kvadrat se potem zapiše kot
\begin{equation} \label{eq:enacba10}
\omega=\frac{2}{1+\sqrt{1- \rho_{Jacobi}^{2}}} \qquad \textrm{ali približno} \qquad \omega= \frac{2}{1+\frac{\pi}{N}}.
\end{equation}
Za bolj splošno obliko dobimo boljšo oceno z nastavkom
\begin{equation}
\omega = \frac{2}{1+ \alpha \frac{\pi}{N}}.
\end{equation} \newpage

Za začetek si poglejmo metodo delovanja na polnem kvadratu (101x101) in trikotniku (201x201). Uporabljena metoda je bila Jacobijeva, pri čemer smo vzeli konstanten $Q_{j,k}$, in sicer:
\begin{equation*}
Q_{j,k}=-q , \qquad q=\frac{1}{\eta} \frac{\Delta p}{\Delta z},
\end{equation*}
pri čemer smo vzeli brezdimenzijsko viskoznost $\eta \approx 1$ za vodo pri 20°C in brezdimenzijski gradient tlaka $\Delta p / \Delta z = -0.28$.
Pomembno je, da z metodo delujemo le na območju, kjer smo na začetku definirali enice. Rezultati za kvadrat so prikazani na slikah \ref{fig:slika3}, začetni pogoji pa \ref{fig:slika2}. Rezultati za trikotnik so predstavljeni na slikah \ref{fig:slika5}, začetni pogoji na sliki \ref{fig:slika4}

\begin{figure}[H]
\begin{center}
\includegraphics[width=0.5\textwidth]{2d_simpson_0.png}
\caption{Hitrostno polje pretoka $u$ pred začetkom postopka.} \label{fig:slika2}
\end{center}
\end{figure}

\begin{figure}[H]
    \centering
    \begin{subfigure}[b]{0.49\textwidth}
        \includegraphics[width=\textwidth]{2d_simpson_1.png}
    \end{subfigure}
    \begin{subfigure}[b]{0.49\textwidth}
        \includegraphics[width=\textwidth]{2d_simpson_2.png}
    \end{subfigure}
    \caption{Hitrostno polje pretoka $u$ po končani iteraciji.} \label{fig:slika3}
\end{figure}

\begin{figure}[H]
\begin{center}
\includegraphics[width=0.45\textwidth]{trikotnik_jacobi_0.png}
\caption{Hitrostno polje pretoka $u$ pred začetkom postopka.} \label{fig:slika4}
\end{center}
\end{figure}

\begin{figure}[H]
    \centering
    \begin{subfigure}[b]{0.45\textwidth}
        \includegraphics[width=\textwidth]{trikotnik_jacobi_1.png}
    \end{subfigure}
    \begin{subfigure}[b]{0.45\textwidth}
        \includegraphics[width=\textwidth]{trikotnik_jacobi_2.png}
    \end{subfigure}
    \caption{Hitrostno polje pretoka $u$ po končani iteraciji.} \label{fig:slika5}
\end{figure}

\noindent Prepričani v dobro delujočo metodo, si poglejmo hitrostno polje pretoka $u$ po izbrani cevi \ref{fig:slika1}. Rešitve in začetne pogoje prikazujejo slike \ref{fig:slika6},  \ref{fig:slika7} in \ref{fig:slika8}. Uporabili smo $N=101$.

\begin{figure}[H]
\begin{center}
\includegraphics[width=0.45\textwidth]{lik_jacobi_0.png}
\caption{Hitrostno polje pretoka $u$ pred začetkom postopka.} \label{fig:slika6}
\end{center}
\end{figure}

\begin{figure}[H]
    \centering
    \begin{subfigure}[b]{0.49\textwidth}
        \includegraphics[width=\textwidth]{lik_jacobi_1.png}
    \end{subfigure}
    \begin{subfigure}[b]{0.49\textwidth}
        \includegraphics[width=\textwidth]{lik_jacobi_2.png}
    \end{subfigure}
    \caption{Hitrostno polje pretoka $u$ po končani iteraciji.} \label{fig:slika7}
\end{figure}

\begin{figure}[H]
\begin{center}
\includegraphics[width=\textwidth]{Figure_4.pdf}
\caption{Hitrostno polje $u$ pretoka .} \label{fig:slika8}
\end{center}
\end{figure}

\noindent Poskusili smo tudi metodo \textit{Gauss-Seidel}, če je konvergenca boljša. Za naš primer pri 101x101 veliki matriki, se izkaže, da sta si metodi enakovredni, pri iskanju rešitve z isto natančnostjo. Metoda \textit{Gauss-Seidel} je bila le za en iterativen korak boljša od metode \textit{Jacobija}.

Sedaj je smiselno poiskati optimalen parameter $\omega$ oz. $\alpha$, ki nastopa v enačbi (\ref{eq:enacba10}). Najprej si poglejmo, kakšen je najbolj smiselen parameter za $N=101$. Število iteracij v odvisnosti od $\omega$ prikazuje slika \ref{fig:slika9}, iz katere razberemo, da je optimalna $\omega_{opt.}=1.85$. Vidimo tudi manjše skoke od $\omega>1.8$. Domnevamo, da je razlog za te skoke premajhno izbrano število $N$.

\begin{figure}[H]
\begin{center}
\includegraphics[width=0.5\textwidth]{omega1.png}
\caption{Število iteracij v odvisnosti od $\omega$ za $N=101$.} \label{fig:slika9}
\end{center}
\end{figure}
\noindent Porodi se vprašanje, če je $\omega_{opt.}(N)$? Ugotovimo, da je. Rezultate prikazuje slednja slika \ref{fig:slika10}. Iz slike razberemo, da je $\omega_{opt.}$ manjša za manjše $N$, kot za večje.

\begin{figure}[H]
\begin{center}
\includegraphics[width=0.5\textwidth]{omega.png}
\caption{Število iteracij v odvisnosti od $\omega$ pri različnih $N$.} \label{fig:slika10}
\end{center}
\end{figure}

Imamo vse sestavine da efektivno poiščemo \textit{Poiseuillov} koeficient za primer $N=101$. Za iskanje smo uporabili $\omega_{opt.}=1.85$. Pretok smo izračunali s pomočjo 2D Simpsonove metode, ki je:
\begin{equation}
\int \int u(x,y) dS \approx \sum _{j,k} W_{j,k} u_{j,k} /N^{2},
\end{equation}
pri čemer je $W$ matrika, kjer imamo za primer 5x5 matriko:
\begin{equation*}
\begin{bmatrix}
1 & 4 & 2 & 4 & 1 \\
4 & 16 & 8 & 16 & 4 \\
2 & 8 & 4 & 8 & 2 \\
4 & 16 & 8 & 16 & 4 \\
1 & 4 & 2 & 4 & 1 
\end{bmatrix}.
\end{equation*} 
Tako z izračuno pretoka izrazimo koeficient $C=1.4165$:
\begin{equation*}
C=\Phi _{V}\frac{8 \pi}{S^{2}}\eta \frac{\Delta z}{\Delta p}, \qquad S= 10/16, \qquad \frac{1}{\eta} \frac{\Delta p}{\Delta z}=0.28.
\end{equation*}
Rezultate prikazujeta sliki \ref{fig:slika11}

\begin{figure}[H]
    \centering
    \begin{subfigure}[b]{0.49\textwidth}
        \includegraphics[width=\textwidth]{poiseuill_1.png}
    \end{subfigure}
    \begin{subfigure}[b]{0.49\textwidth}
        \includegraphics[width=\textwidth]{poiseuill_2.png}
    \end{subfigure}
    \caption{\textit{Poiseuillov} koeficient za primer $N=101$.} \label{fig:slika11}
\end{figure}

\noindent Z večanjem števila $N$, bi tudi povečali koeficient, saj bi s tem zapolnili vedno več praznega prostora, ki ga z mnajšimi $N$ nismo mogli. To povečanje ni veliko in bi se v primeru $N=201$ poznalo šele na drugi ali tretji decimalki. Z dodatnim večanjem bi se pričele spreminjati decimalke na višjh številih.

\section*{Temperaturni profil valja}

Imamo enakostraničen kovinski valj ($2r=H$), ki ga grejemo na spodnji ploskvi s konstantnim toplotnim tokom, plašč prerežemo horizontalno na dve enako široki polovici pri različnih temperaturah, zgornjo ploskev pa držimo na isti temperaturi, kot spodnjo polovico plašča. Določiti želimo temperaturni profil.

Reševanje se v tem primeru ne spremeni veliko. Iz kartezičnih koordinat preidemo v cilindrične. Za temperaturni profil v stacionarnem stanju velja \textit{Laplace-ova} enačba
\begin{equation*}
\nabla ^{2} u=0,
\end{equation*}
toda izpeljali bomo za bolj splošno \textit{Poisson-ovo} enačbo
\begin{equation*}
\nabla ^{2} u=-q.
\end{equation*}
\textit{Poisson-ova} enačba se v cilindričnih koordinatah glasi:
\begin{equation}
\nabla ^{2} u=  \frac{\partial ^{2} u}{ \partial r^{2}} + \frac{1}{r} \frac{\partial u}{\partial r} + 
\frac{1}{r^{2}}\frac{\partial ^{2} u}{\partial \theta ^{2}} + \frac{\partial ^{2} u}{\partial z^{2}}=-q,
\end{equation} 
pri čemer so:
\begin{align}
& \frac{\partial ^{2} u}{\partial r^{2}} =\frac{u_{j,k-1} - 2 u_{j,k} + u_{j,k+1}}{h^{2}}, \\
& \frac{\partial ^{2} u}{ \partial z^{2}} =  \frac{u_{j-1,k} - 2 u_{j,k} + u_{j+1,k}}{h^{2}}, \\
& \frac{1}{r} \frac{\partial u}{\partial r} =  \frac{1}{r_j} \frac{u_{j,k+1} - u_{j,k-1}}{2h}, \\
& \frac{1}{r^{2}}\frac{\partial ^{2} u}{\partial \theta ^{2}}= 0.
\end{align}
Če sedaj delujemo po zgledu enačb (\ref{eq:enacba5}) in (\ref{eq:enacba6}), upoštevanjem $h= \Delta x = \Delta y$ in $d=4/h^{2}$, dobimo \textit{Jacobi-jevo} iteracijsko shemo za cilindrične koordinate
\begin{align} \label{eq:enacba17}
\tilde{u}^{n}_{j,k}=u^{n+1}_{j,k}= \frac{1}{d} &\left[ Q_{j,k} \right. \nonumber \\
+& \frac{u_{j-1,k}^{n} + u_{j+1,k}^{n} + u_{j,k-1}^{n}  + u_{j,k+1}^{n} }{h^{2}} \nonumber \\
+ &\left.  \frac{1}{r_k} \frac{u_{j,k+1}^{n} - u_{j,k-1}^{n}}{2h} \right].
\end{align}
V splošnem rešujemo problem na enak način kot prej
\begin{equation*}
u_{j,k} ^{n+1} = u_{j,k} ^{n} + \omega \left( \tilde{u}_{j,k} ^{n} - u_{j,k} ^{n}  \right).
\end{equation*}

V enačbi (\ref{eq:enacba17}) nastopa člen $1/r_j$, ki povzroča težave, saj je singularen v točki 0. Problema se rešimo tako, da definiramo $r_j$ izmaknjen iz ihodišča:
\begin{align*}
r_k= -1+ h\left( j + 1/2 \right), \qquad j=0,1,2, ...   N-1
\end{align*}
Ker je reševanje simetrično glede na $r$, se bomo poslužili trika in problem reševali s pomočjo zrcaljenja. Tako bomo za vsak $r_j$ zapisali še $r_{-j}$ in dobili $2N$ enakomerno razporejenih točk. Robne pogoje bomo uporabili enake na $r_{-N}=r_{N}$.  V $z$ smeri je stvar trivialna. $z$ razdelimo na $2N$ točk.

Robne pogoje za konstantno temperaturo poznamo. Spremenijo pa se robni pogoji, pri katerih imamo konstanten odvod (konstanten toplotni tok $\rightarrow$ grelec na dnu valja ali izoliran rob).
Tako imamo pogoje glede na rob:
\begin{align} \label{eq:enacba18}
& \frac{\partial u}{\partial r}\biggr|_{\partial R} = \frac{u_{j,k+1} - u_{j,k-1}}{2h} = K_1 \nonumber \\
& \rightarrow u_{j,k+1}= 2hK_1 + u_{j,k-1} \qquad \textrm{ali} \qquad u_{j,k-1}= u_{j,k+1} -2hK_1
\end{align}
in
\begin{align} \label{eq:enacba19}
& \frac{\partial u}{\partial z}\biggr|_{\partial R} = \frac{u_{j+1,k} - u_{j-1,k}}{2h} = K_2 \nonumber \\
& \rightarrow u_{j+1,k}= 2hK_2 + u_{j-1,k} \qquad \textrm{ali} \qquad u_{j-1,k}= u_{j+1,k} -2hK_2
\end{align}
Enačbi (\ref{eq:enacba18}) in (\ref{eq:enacba19}) vstavimo v \textit{Jacobi-jevo} iteracijsko shemo (\ref{eq:enacba17}) glede na zahtevan robni pogoj in dobimo iteracijsko shemo za rob našega iskanega območja. Povsod drugje uporabimo originalno enačbo (\ref{eq:enacba17}).

Imamo vse sestavine, da si pogledamo iskani problem. Za začetek si poglejmo primer, kjer grejemo spodnjo ploskev s konstantnim toplotnim tokom. To nam da robni problem za enačbo (\ref{eq:enacba19}), kjer določimo koeficient $K_1=0.5$. Ostali robovi so na konstantni temperaturi 0 ali 2. Rešitev prikazuje slika \ref{fig:slika12}.

\begin{figure}[H]
\begin{center}
\includegraphics[width=0.65\textwidth]{valj3.png}
\caption{Rešitev 1.} \label{fig:slika12}
\end{center}
\end{figure}

\noindent Naloga zahteva, da si pogledamo še primer, kjer zgornja polovica plašča nima več konstantne temperature, temveč je izolirana. Robni pogoj poiščemo iz enačbe (\ref{eq:enacba18}), kjer velja za izoliran rob $K_1=0$. Ostali pogoji so ostali enaki. Rešitev je prikazana na sliki \ref{fig:slika13}.

\begin{figure}[H]
\begin{center}
\includegraphics[width=0.65\textwidth]{valj4.png}
\caption{Rešitev 2.} \label{fig:slika13}
\end{center}
\end{figure}

Kot dodatek si poglejmo še, kako je, če odvajamo toploto $\rightarrow$ spremenimo $K_2=-0.5$. Še vedno imamo izoliran zgornji plašč, ostale robove držimo na temperaturi 2. Rešitev je prikazana na sliki \ref{fig:slika14}.

\begin{figure}[H]
\begin{center}
\includegraphics[width=0.65\textwidth]{valj5.png}
\caption{Rešitev 3.} \label{fig:slika14}
\end{center}
\end{figure}


\end{document}
